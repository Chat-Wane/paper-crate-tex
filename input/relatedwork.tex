\section{Related Work}
\label{sec:relatedwork}

Optimistic replication is a replication strategy where each peer owns a local
copy (or replica) of the shared data. Since peers can directly modify this copy,
it improves both availability and responsiveness of the data. When a peer
performs an operation on the local data, the information is spread to all owners
of the shared data where they are integrated. Optimistic replication approaches
guarantee eventual consistency~\cite{bailis2013eventual},i.e., after the
receipt and integration of all operations by the members, the replicas converge
to an identical state. This section introduces the shared data structure that
guarantees eventual consistency and the dissemination protocol.

To insert a character in a shared document, one cannot just use the standard
well-known operation \emph{insert element at index} of arrays and expect
guaranties on convergence. For instance, if two users concurrently edit a
sequence ERTY: the first user inserts a Q at index 0, while the second user
inserts a W at index 0. They respectively obtains QERTY and WERTY as resulting
sequence. On receipt of each other's operation they respectively obtain WQERTY
and QWERTY. Hence the need to inspect these concurrent cases to ensure
convergent replicas.

Currently, most of web collaborative editors (REF) use operational transform
(OT) approaches whose principle lies in modifying the arguments of the received
operation against its concurrent ones. For instance, in the prior example, when
the first user receives \emph{insert W at 0}, it detects \emph{insert Q at 0} as
concurrent, and decides to transform the operation into \emph{insert W at 1}.
In that spirit, a wide variety of OT approaches exist for text editing, image
editing etc. In text editing context, OT can provide the usual insertion and
deletion operations plus a range of string-wise operations such as move,
cut-paste, etc. However, operational transform approaches have drawbacks. First,
the quadratic time complexity of the transformations compared to the number of
concurrent operations which precludes its usage on the Internet subject to
unpredictable latency and, in particular, the existence of an offline editing
mode. Second, the structure to detect concurrency is expansive, i.e., it
requires the history of all operations and a causality tracking
mechanism. \TODO{Moar}.

In order to alleviate these issues, web editors communicate through a central
server. Thus, each operation transits through the server which transforms them
and forwards them to all connected participants. However, the topology itself
lacks of scalablity and brings privacy issues, censorship issues,
single-point-of-failure, economic intelligence issues, and restriction in terms
of service. In other terms, the shared document does not fully belong to its
authors since the owner of the server controls it.

%%% Local Variables: 
%%% mode: latex
%%% TeX-master: "../paper"
%%% End: 
