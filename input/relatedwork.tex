\section{Related Work}
\label{sec:relatedwork}

Real-time distributed collaborative editors consider multiple participants, each
hosting a copy -- or replica -- of a shared sequence of characters. The
participants can update their copy by inserting or deleting characters at any
time. Then, the editor eventually delivers the change to all
participants. Finally, each editor integrates the received
operations~\cite{saito2005optimistic}. Strong eventual
consistency~\cite{bailis2013eventual, shapiro2011comprehensive,
  sun1998achieving} requires that editors integrating a same set of operations
converge to an equivalent state, i.e. users see the same document. Furthermore,
operations must preserve the intention of users, i.e. effects observed at
generation time must be re-observed at integration time regardless of concurrent
operations. Finally, the integration order of operations must follow the
\emph{happen before} relationship~\cite{lamport1978time} stated at generation.

\noindent Four complexities characterize editors:
\begin{itemize}[noitemsep, leftmargin=*]
\item Generation time: complexity to generate locally an operation.
\item Integration time: complexity to execute remotely an operation.
\item Space complexity: complexity to store a local copy of the shared sequence.
\item Communication complexity: complexity of messages transiting the network.
\end{itemize}
Solving scalability issues requires finding a balance between communication,
space and time complexities %  Among others, the communication complexity is the
% most discriminant and depends of metadata required to ensure strong eventual
% consistency as well as messages dissemination in the network.
where the communication complexity plays the most important role. The challenge
consists in providing a sublinear upper bound on communication complexity
without making other complexities impracticable.

% It requires $\mathcal{O}(m.\ln{R})$ where ($\ln{R}$) is a minimal multiplicative
% factor induced by the broadcasting mechanism which depends of the session size
% $R$ including both readers and writers; where $m$ is the message embedding
% metadata necessary to maintain consistency. To scale, the space complexity of
% these metadata must be sublinear.


Operational transformation (OT) allows building both centralized and
decentralized real-time editors. At generation, the processing time of
operations is constant. At integration, OT transforms the received operations
according to concurrent ones which were generated on the same state. An
integration algorithm such as COT~\cite{sun2009contextbased} or
SOCT2~\cite{vidot2000copies} along with transformation functions (ensuring
transformation properties) guarantee a consistent model. The integration time
depends on concurrency and differs among the algorithms.  For instance, COT's
integration time complexity is exponential. It can be reduce to linear at the
expense of space complexity. SOCT2's integration time complexity is quadratic in
terms of number of concurrent operations.  Whatever the tradeoff, OT's
integration algorithms rely on concurrency detection which costs, at least, a
state vector~\cite{charronbost1991concerning} the size of which grows linearly
compared to the number of members $W$ who ever participated in the
authoring. Since each message carries such a vector, decentralized OT approaches
are not practicable in large groups subject to churn where people join and leave
the network frequently and freely. % As for space complexity, OT
% approaches require an historic of operations $H$, each operation being linked
% to their state vector, hence $\mathcal{O}(W.H)$. Such monotonically growing
% historic can be cut at the price of consensus which, once again, does not
% scale~\cite{mostefaoui2015signature}. Thus, OT approaches are the best in
% humble confined environment like local area networks. However, their
% performance degrades when the network size grows and becomes unpredictable
% like wide area networks.

Conflict-free replicated data types (CRDTs) for
sequences~\cite{shapiro2011comprehensive, shapiro2011conflict}] constitute the
latter approaches which solve concurrent cases by providing commutative,
associative, and idempotent operations. As such, and compared to OT, the
causality tracking cost is drastically reduced since it only requires tracking
semantically related operations. For instance, the removal of a particular
element must follow its insertion. The commutative property of operations
ensures the consistency of the system. CRDTs provide commutativity by
associating a unique and immutable identifier to each element. Defining a total
order among the identifiers allows retrieving the sequence of characters.

\noindent CRDTs propose an interesting tradeoff since they can balance complexities
depending on the type of structure that represents the document.  In particular,
increasing the generation time of operations to decrease the integration time is
profitable since an operation generated once is re-executed many
times. Nevertheless, the identifiers consume space which, in turns, impacts the
communication complexity.

% WOOT, WOOT, WOOTH, RGA, SW, PPS, DiCE
Tombstone-based CRDTs~\cite{ahmed2011evaluating, conway2014language,
  grishchenko2010deep, oster2006data, roh2011replicated, weiss2007wooki,
  wu2010partial, yu2012stringwise} such as WOOT~\cite{oster2006data} associate a
constant size identifier $\mathcal{O}(1)$ to each element but removals of
elements only hide them to the users. Therefore, removed elements keep consuming
space leading to a monotonically growing document, hence $\Theta(I)$ where $I$
is the number of insertions performed on the document.  Destroying removed
elements requires running consensus algorithms to determine if everyone received
the removal and agree on definitely throw out the element. Such algorithms are
prohibitively costly and do not scale in terms of number of members, especially
in networks subject to churn~\cite{mostefaoui2015signature}. Since the structure
contains the full history of the document, such an approach does not require
further information to track causally related operations.

Variable-size identifiers CRDTs~\cite{nedelec2013lseq, preguica2009commutative,
  weiss2009logoot} such as Logoot~\cite{weiss2009logoot} truly destroy elements
targeted by removals.  Since removals truly erase information from the
structure, these approaches require a local state vector compacting their
history, hence an additional $\mathcal{O}(W)$ on space complexity. It is worth
noting that, contrarily to OT, the vector is only stored locally. Variable-size
identifiers approaches allocate identifiers the size of which is determined at
generation. The allocation function becomes crucial to maintain identifiers
under acceptable boundaries. Unfortunately, they depend on the insert position
of elements. For instance, writing the sequence QWERTY left-to-right allocates
the identifiers [1] to Q, [2] to W, [3] to E, \ldots, [6] to Y. But with an
identical strategy, writing the same sequence right-to-left allocates the
identifiers [1] to Y, [1.1] to T, [1.1.1] to R, \ldots We observe a quick growth
of identifiers depending on the editing behavior. In both cases, the growth is
linear compared to the number of insertions $I$ in the sequence,
i.e. $\mathcal{O}(I)$. Both Treedoc~\cite{preguica2009commutative} and
Logoot~\cite{weiss2009logoot, weiss2010logootundo} suffer from the growth of the
identifiers which, in turn, impacts the generated traffic.
% If the structure stores each identifier in a flat array, it grants fast
% access
% at the price of quadratic space complexity $\mathcal{O}(I^2)$. If it
% factorizes common parts of identifiers as a tree structure, it achieves a
% linear space complexity.
To provide small identifiers, these approaches eventually require balancing the
structure, i.e., relocating identifiers. This requires a global agreement which is
akin to run a distributed consensus
protocol~\cite{zawirski2011asynchronous}. Unfortunately, these protocols do not
scale.

\noindent The algorithm \LSEQ~\cite{nedelec2013lseq} aims to avoid such
consensus by sublinearly upper-bounding the space complexity of variable-size
identifiers. It is conjectured to have a polylogarithmic progression of its
identifiers size $\mathcal{O}((\log I)^2)$ compared to the number of insertions
in the document. This paper proves the complexity upper bounds and states the
conditions under which it applies.


%%% Local Variables: 
%%% mode: latex
%%% TeX-master: "../paper"
%%% End: 
