\section{Related Work}
\label{sec:relatedwork}

\begin{table*}
  \centering
  

\begin{tabular}{@{}lccccccc@{}}
  \toprule
  & \multicolumn{4}{c}{\textsc{time}} & \multicolumn{2}{c}{\textsc{space}} & \textsc{communication} \\ \cmidrule{2-8}
  & \multicolumn{2}{c}{\textsc{local}} & \multicolumn{2}{c}{\textsc{remote}} & & & \\ \cmidrule{2-5}
  & \textsc{ins} & \textsc{del} & \textsc{ins} & \textsc{del} & \textsc{sequence} & \textsc{causality} & \textsc{message} \\ \midrule
  SOCT2/TTF & $O()$ & $O()$ & $O()$ & $O()$ & $O()$ & $O()$ & $O()$ \\ \midrule
  \TODO{COT-DO} & $O()$ & $O()$ & $O()$ & $O()$ & $O()$ & $O()$ & $O()$ \\ \midrule
  WOOT & $O()$ & $O()$ & $O()$ & $O()$ & $O()$ & $O()$ & $O()$ \\ \midrule
  WOOTO & $O()$ & $O()$ & $O()$ & $O()$ & $O()$ & $O()$ & $O()$ \\ \midrule
  WOOTH & $O()$ & $O()$ & $O()$ & $O()$ & $O()$ & $O()$ & $O()$ \\ \midrule
  RGA & $O()$ & $O()$ & $O()$ & $O()$ & $O()$ & $O()$ & $O()$ \\ \midrule
  \TODO{SW} & $O()$ & $O()$ & $O()$ & $O()$ & $O()$ & $O()$ & $O()$ \\ \midrule
  \TODO{PPS} & $O()$ & $O()$ & $O()$ & $O()$ & $O()$ & $O()$ & $O()$ \\ \midrule
  \TODO{Neil Conway} & $O()$ & $O()$ & $O()$ & $O()$ & $O()$ & $O()$ & $O()$ \\ \midrule
  Treedoc & $O()$ & $O()$ & $O()$ & $O()$ & $O()$ & $O()$ & $O()$ \\ \midrule
  Logoot & $O()$ & $O()$ & $O()$ & $O()$ & $O()$ & $O()$ & $O()$ \\ \midrule
  LogootSplit & $O()$ & $O()$ & $O()$ & $O()$ & $O()$ & $O()$ & $O()$ \\ \midrule
  \TODO{\LSEQ}  & $O()$ & $O()$ & $O()$ & $O()$ & $O()$ & $O()$ & $O()$ \\ \bottomrule
\end{tabular}


  \caption{\label{table:complexities}Upper-bound ($\mathcal{O}$) on complexities in time, space, and communication of decentralized approaches.}
\end{table*}

\begin{asparadesc}
\item[Optimistic replication] is a replication strategy where each peer owns a
  local copy (or replica) of the document. Since users can directly modify this
  copy, it improves both availability and responsiveness. When a user performs
  an operation on the local document, the information is spread to all owners of
  the shared document where they are integrated. Optimistic replication
  approaches guarantee eventual consistency~\cite{bailis2013eventual}. Hence,
  after the receipt and integration of all operations by the participants, the
  replicas converge to an identical state.
\item[Centralized] collaborative editors constitute an easy way to ensure
  documents eventually consistent. Until recently, they suited very well the web
  context since browsers were limited to the traditionnal client-server
  paradigm. Therefore, there exist a plethora of these centralized editors
  (e.g. Google Docs, Etherpad, Apache Wave).  Nevertheless, the topology itself
  suffers issues in privacy, censorship, and economic intelligence.  Indeed,
  since all communications transit a server, the owner of the latter is able to
  intercept and use the information without the users' consent (\TODO{REF}).
  Additionally, centralized solution lacks of scalability and ease of
  deployment. People who wants to build their own document provider must set a
  server and carefully check the latter's capacity to hold the load of users
  (\TODO{REF}).  Finally, users must agree to restriction in terms of service.
  For instance, Google Docs restricts the writing access of its documents to 50
  users simultaneously. Beyond, the users are readers (\TODO{REF}). \TODO{Such
    limitations preclude the existence of crowd edited documents which could be
    useful in educational cases.}
\item[Decentralized] collaborative editors are uncommon comparatively. Yet, the
  appearance of peer-to-peer connections from browser-to-browser opens the field
  to such editors which become directly accessible in end-user web
  browsers. Consequently, they are as practicable as centralized web editors and
  solve all the latter's aforementioned issues except the scalability. Indeed,
  with such topology, the algorithms and data structures are rethought to handle
  the operations arriving disorderly.  To insert a character in a shared
  document, one cannot just use the standard well-known operation \emph{insert
    element at index} of arrays and expect guaranties on convergence. For
  instance, if two users concurrently edit a sequence ERTY: the first user
  inserts a Q at index 0, while the second user inserts a W at index 0. They
  respectively obtains QERTY and WERTY as resulting sequence. On receipt of each
  other's operation they respectively obtain WQERTY and QWERTY. Hence the need
  to inspect these concurrent cases to ensure convergent replicas.  Many
  approaches have been proposed with the challenging task to find a good
  compromise between complexities in time, space, and communication.
\item [Operational transformation] constitute the former approaches aiming to
  solve the concurrent cases. Their principle lies in modifying the arguments of
  the received operation against its concurrent ones. For instance, in the prior
  example, when the first user receives \emph{insert W at 0}, it detects
  \emph{insert Q at 0} as concurrent, and decides to transform the operation
  into \emph{insert W at 1}.  In that spirit, a wide variety of OT approaches
  exist for text editing, image editing etc. In text editing context, OT can
  provide the usual insertion and deletion operations plus a range of
  string-wise operations such as move, cut-paste, etc. However, operational
  transform approaches have drawbacks. \TODO{First, the quadratic time
    complexity of the transformations compared to the number of concurrent
    operations which precludes its usage on the Internet subject to
    unpredictable latency and, in particular, the existence of an offline
    editing mode. Second, the structure to detect concurrency is expansive,
    i.e., it requires the history of all operations and a causality tracking
    mechanism}. \TODO{moar}.
\item [Conflict-free replicated data types] (CRDTs) for sequences constitute the
  latter approaches which solve the concurrent cases by providing commutative
  and idempotent operations. Thus, the receipt order of operations does not
  impact the final resulting document. These approaches provide commutativity by
  associating a unique and immutable identifier to each element. Defining a
  total order among the identifiers allows retrieving the sequence. There exist
  two class of CRDTs for sequences exposing different tradeoff between their
  complexities.
\item [Tombstone-based] CRDTs (\TODO{REF}) which associate a constant size
  identifier to each element but whose removals of elements only hide them to
  the users. Therefore, removed elements keep consumming space and impact on
  performance. For instance, documents like Wikipedia pages which are subject to
  vandalism and edit wars (\TODO{REF}) could appear empty while weighting
  heavy. Such approaches eventually need a garbage collecting mechanism
  (\TODO{REF}).  Unfortunately, decentralized garbage collecting algorithms
  require consensus (\emph{does everyone see the removal and agree on definitely
    throw out the element}) which is prohibitevely costly and does not scale in
  number of users, especially in network subject to churn (where members join
  and leave the network frequently and freely).
\item [Variable-size identifiers] CRDTs whose removals truely destroy the
  targeted elements but whose identifiers are allocated with different size at
  generation. In these approaches, the allocation function is crucial to
  maintain identifiers under acceptable boundaries. Unfortunately, they depend
  of the insert position of elements. For instance, writing the sequence QWERTY
  left-to-right allocates the identifiers [1] to Q, [2] to W, [3] to E, \ldots,
  [6] to Y. But with an identical strategy, writing the same sequence
  right-to-left allocates the identifiers [1] to Y, [1.1] to T, [1.1.1] to R,
  \ldots As we observe, depending on the editing behaviour, the identifiers
  can grow quickly. \TODO{moar}.
\item [As summary] Table~\ref{table:complexities} shows the upper-bound
  ($\mathcal{O}$) of complexities in space, time, and communication of
  decentralized approaches. In this table, we can see that both operationnal
  transformations approaches, namely SOCT2/TTF and COT-DO, do not scale in
  communication since they send messages that grows linearly compared to the
  number of writers. Conversely, the representatives of tombstone-based CRDTs
  (WOOT, WOOT, WOOTH, RGA, SW, PPS, Neil) communication complexity only depends
  of the dissemination protocol (in logarithm of the number of replicas),
  however their space and time complexities include the removals. In other
  terms, these approaches are monotonically growing structure, hence, they do
  not scale in number of operations performed on the document. The variable-size
  identifiers CRDTs comprise Treedoc, Logoot, and \LSEQ. Their complexities
  depend of insert positions (or editing behaviour) of the elements. Treedoc, as
  a hybrid solution, uses tombstones and suffers of the aforementionned
  issue. Like Logoot, it mainly targets monotonic editing at the end of the
  sequence. Unfortunately, when the editing behaviour does not comply with this
  assumption, the space complexity grows quadratically. \LSEQ provides a
  sublinear upper-bound on its space complexity. Thus, it scales in document
  size, and communications. Yet, it requires a local causality tracking
  mechanism increasing linearly compared to the number of
  writers. \TODO{cf. proposal?}.
\end{asparadesc}

%%% Local Variables: 
%%% mode: latex
%%% TeX-master: "../paper"
%%% End: 
