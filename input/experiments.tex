\newcommand{\FIGUREWIDTH}{0.7}

\section{Experiments}
\label{sec:experiments}

The objective of this experiment section is twofold:
\begin{itemize}
\item To confirm the complexity of \LSEQ. Monotonic editing behavior should lead
  \begin{inparaenum}[(i)]
  \item to a polylogarithmic growth of identifiers compared to the number of
    insertions;
  \item to a logarithmic increasing of execution time except for the lookup
    which is expected to be linear.
  \end{inparaenum}
\item To show the impact on the generated traffic on a real decentralized
  collaborative editor. We expect that both the identifier size and neighborhood
  size impact the traffic. Since the former grows polylogarithmically, and the
  latter grows logarithmically, we expect the traffic to scale in terms of
  the number of users and the number of operations.
\end{itemize}


\subsection{Space complexity}
\label{subsec:space}

\begin{asparadesc}
\item [Objective:] To confirm the space complexity of \LSEQ's identifiers and to
  highlight the improvement over state-of-the-art.
\item [Description:] We consider two editing behaviors:
  \begin{inparaenum}[(i)]
  \item random which consists in inserting elements at random position in the sequence;
  \item left-to-right which consists in inserting elements at the end of the
    sequence.
  \end{inparaenum} We measure the average bit size of paths allocated by
  Treedoc~\cite{preguica2009commutative}, Logoot~\cite{weiss2009logoot}, and
  \LSEQ~\cite{nedelec2013lseq}. The generated documents reach half a million
  characters.

\begin{figure}
  \centering
  \includegraphics[width=\columnwidth]{./img/space.eps}
  \caption{\label{fig:space}Size of paths allocated by Treedoc, Logoot, and
    \LSEQ.}
\end{figure}

\item [Result:] Figure~\ref{fig:space} shows the results of the experiment. The
  x-axis denotes the size of the document. The y-axis denotes the average size
  of generated paths. The top figure shows the measurements of the random
  editing behavior while the bottom figure shows the measurements of the
  left-to-right editing behavior. As expected, we observe that random editing
  leads to logarithmic growth of paths whatever the allocation function. In
  addition, Treedoc allocates the smallest paths, followed by \LSEQ, followed by
  Logoot. Nonetheless, on left-to-right editing, paths allocated by Treedoc grow
  extremely quickly to such an extent that its plot merges with the
  y-axis. Logoot's paths grow slower but still linearly compared to the number
  of insertions in the document. We observe that \LSEQ exposes a better space
  complexity compared to the state-of-the-art, which eventually leads to smaller
  allocated paths.
\item [Reason:] The random editing leads to a balanced underlying structure,
  hence the logarithmic progression of paths allocated by the three
  strategies. In this case, Treedoc is better because it has an arity of $2^1$,
  for it is a binary tree. In this setup, \LSEQ starts with an arity of $2^8$
  while Logoot has an arity of $2^{16}$, hence \LSEQ allocating smaller paths than
  Logoot in random editing. The left-to-right editing behavior tends to
  unbalance the tree over insertions. In Treedoc, monotonic editing is the worst
  case scenario where each new path contains all paths allocated before
  it. Logoot is designed to handle left-to-right editing. It allocates paths but
  keeps space for upcoming insertions. Still, since its maximum arity is
  constant, the number of characters in a branch is constant, hence the linear
  growth of paths. On the opposite, \LSEQ doubles its arity at each level of the
  tree. The branch can store twice as much elements as its parent. Hence a
  growth that scales sublinearly compared to the number of insertions in the
  document.
\end{asparadesc}

\subsection{Time complexity}


\begin{asparadesc}
\item [Objective:] To confirm the time complexity of \LSEQ's operations. We
  expect good scalability of operations except for the lookup after monotonic
  editing.
\item [Description:] This experiment involves one user who performs operations
  on its local copy of the document. The benchmark ran on a MacBook Pro with 2.5
  GHz Intel Core i5, with Node.js 4.1.1 on Darwin 64-bit. For each operation, we
  create a document containing $I-1$ characters and measure the time taken by
  the $I^{th}$ operation. The operation set includes the lookup, the local part
  of an insertion, the remote part of an insertion, and the remote part of a
  deletion -- the local part of a deletion only consist of broadcasting the
  identifier. We perform the measurements multiple times on two kinds of
  documents. Firstly, a document generated by random editing, i.e. the
  underlying \LSEQ tree is balanced. Secondly, a document generated by monotonic
  editing, i.e. one branch per level of the \LSEQ tree is filled.  We focus on
  tendencies rather than absolute values. Javascript performs on-the-fly
  optimization which we limit in order to show the real time contribution of
  each operation.

\begin{figure}
  \centering
  \includegraphics[width=\columnwidth]{./img/time.eps}
  \caption{\label{fig:time} Performance of the umptieth operation performed in
    \LSEQ.}
\end{figure}

\item [Result:] Figure~\ref{fig:time} shows the result of this experiment. The
  x-axis denotes the number of insert operations performed before the measured
  operation. The y-axis denotes the average time taken by this operation. Both
  axis are on a logarithmic scale. The top part of the figure focuses on a
  structure filled with insertions following a random editing behavior while the
  bottom part of the figure focuses on a structure filled with insertions
  following a monotonic editing behavior. We observe that the measured values
  barely grow after random editing, regardless of the operation. On the other
  hand, while the local insertion after monotonic editing remains stable, we
  observe a linear growth with the lookup execution time, and a slower growth
  for both the remote insertions and the remote deletions.
\item [Reason:] After the insertions at random positions, the underlying tree of
  \LSEQ is balanced. Therefore, the range of influence of each operation is
  limited to a small subset of the elements composing the document. For
  instance, the lookup operation does not need to explore each element of the
  tree. Instead, it quickly discards a lot of irrelevant branches at each level
  of the tree because the index does not fall into their range. However, this
  remark does not hold when the insertions followed a monotonic editing
  behavior. Indeed, in this case, most elements are located in the one and
  deepest level of the tree. Thus, the lookup likely crawls to this level and
  then inspects each elements to count their children and actualize its current
  index. The remote operations measurements follow the same reasoning: they must
  perform a binary search at each depth of the tree; since the random editing
  structure is balanced, the average depth of the search is smaller compared to
  the structure resulting from monotonic editing behavior. Hence, a lower time
  complexity.
\end{asparadesc}

\subsection{Communication complexity}

\begin{asparadesc}
\item [Objective:] To show that our \LSEQ-based decentralized collaborative
  editor scales well in terms of number of participants and document size. We
  expect the traffic to grow polylogarithmically compared to the document size
  (contribution of \LSEQ), and logarithmically compared to the number of
  participants (contribution of \SPRAY).
\item [Description:] This experiment was conducted on the Grid'5000 testbed with
  101 machines running from 1 to 6 web browsers. The results concern editing
  sessions involving from 101 members to 601 members. The members of each
  session collaboratively write a document by repeatedly inserting new
  characters at the end of the document. We measure the average outgoing traffic
  of members, the average size and variance of partial views. An arbitrary
  number of 100 operations per second are uniformly distributed among
  participants. The editing sessions last 7 hours. The documents reach millions
  of characters.

\begin{figure}
  \centering
  \includegraphics[width=\columnwidth]{./img/communication.eps}
  \caption{\label{fig:traffic} Average traffic per second at each editor during
    real-time editing sessions.}
\end{figure}

\item [Result:] Figure~\ref{fig:traffic} shows the result of this experiment.
  The x-axis denotes the experiment progression. The y-axis shows the average
  outgoing traffic generated by members. The legend shows the average size and
  variance of partial views for each editing session.  As expected, we observe
  two results:
  \begin{inparaenum}[(i)]
  \item the growth of each individual plot corresponds to a polylogarithmic
    increasing due to the paths allocated by \LSEQ;
  \item the growth between plot corresponds to a logarithmic increasing due to
    the partial views maintained by \SPRAY.
  \end{inparaenum}
  In addition, the load is balanced among participants thanks to \SPRAY.  Hence,
  \CRATE scales well in terms of number of participants and number of
  insertions. Treedoc- and Logoot-based collaborative editors would have linear
  growth, for both Treedoc and Logoot allocate linearly growing identifiers
  during left-to-right editing (as shown in Section~\ref{subsec:space}).
\item [Reason:] When an outsider joins the network, \SPRAY injects a logarithmic
  number of connections compared to the network size. Since it does not use any
  global knowledge, this number is a rough estimation and may lead to a small
  deviation between the actual value and the expected theoretical value. For
  instance, the editors of the editing session involving 601 members have a
  smaller view in average than the editors of the editing session involving 501
  members (6.3 against 6.7). As consequence, the generated traffic of the former
  is smaller than the generated traffic of the latter. Nevertheless, regardless
  of the editing session, the traffic is well balanced among participants, for
  the variance in view sizes remains small.

  Each time an insertion is performed, an identifier is created and broadcast.
  Each editor sends each identifier to all the editors included in its partial
  view exactly once. Since the identifiers grow polylogarithmically compared to
  the number of insertions (thanks to \LSEQ), and since the views grow
  logarithmically compared to the network size (thanks to \SPRAY), the resulting
  traffic is the product of both this polylogarithm and this logarithm.
\end{asparadesc}

%%% Local Variables:
%%% mode: latex
%%% TeX-master: "../paper"
%%% End:
