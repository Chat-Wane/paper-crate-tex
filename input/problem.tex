\section{Abstract problem}
\label{sec:problem}
To highlight the problem, we withdraw the unnecessary aspects inherent to CRDTs
and distributed collaborative editing. The ``From Mutable to Immutable'' simply
depicts the problem as a incremental translation from a set where the indices
are mutable, (i.e., an element is linked to a position that may change) to a
set where indices are immutable.
\ \\ \ \\
\textbf{[From Mutable to Immutable problem]} Let $X$ the goal sequence composed
of elements from an alphabet $\mathcal{A}$, $Y$ the sequence resulting of the
permutation of $X$ and containing additional elements from a set $\mathcal{M}$
equipped with a total order $(\mathcal{M},\, <_\mathcal{M})$, $Z$ the set
containing the elements of $X$ and immutable elements from a set $\mathcal{I}$
equipped with a dense total order $(\mathcal{I},\,
<_\mathcal{I})$.\\ \ \\ Given:
\vspace{-\topsep}
\begin{itemize} \itemsep0em
\item $X: \mathbb{N} \rightarrow \mathcal{A}$
\item $Y: \mathbb{N} \rightarrow \mathcal{A} \times \mathcal{M}$
\item $Z: \mathbb{N} \rightarrow \mathcal{A} \times \mathcal{I}$
\item $gen\beta:\mathcal{M} \rightarrow \mathcal{I}$
\item $Z$-uniqueness:\\ $\forall\langle \alpha_1,$ $\beta_1
  \rangle$, $\langle \alpha_2,\beta_2 \rangle \in Z$, $\beta_1
  = \beta_2 \Rightarrow \alpha_1 = \alpha_2$
\item $Z$-order-preservation: $Z(i) = \langle \alpha_i,\, \beta_i\rangle
  \Rightarrow X(i) = \alpha_i$
\item $Z$-specification: ($Z:Y\times\mathbb{N}\rightarrow Z$):
  \vspace{-\topsep}
  \begin{itemize} \itemsep0em
  \item $Z(\varnothing,\,\_\,) \rightarrow \varnothing$
  \item $Z(\_\,,\, 0) \rightarrow \varnothing$
  \item $Z(Y,\, i) \rightarrow$ Let $Y(i) = \langle \alpha_i,\,\mu_i \rangle$:
    $Z(Y,\, i-1) \cup \{\alpha_i,\, gen\beta(\mu_i) \}$
  \item $Z(Y,\,|Y|)$
  \end{itemize}
\end{itemize}
The problem is to find an optimal function $gen\beta$ such that there do not
exist any functions $gen\beta'$ such that for any permutation $Y$, the
resulting set $Z'$ using $gen\beta'$ has a lower binary representation than the
resulting set $Z$ using $gen\beta$.
\ \\

Put back in the distributed collaborative editing context, the goal sequence
$X$ corresponds to the intention of authors, i.e., the final document that
peers will eventually write (e.g., $QWERTY$ in the paper). The sequence $Y$ is
an editing sequence performed by the authors to reach that goal (e.g.,
$[(Q,\,0)$, $(W,\,1)$, $\ldots]$). However, using such indices to order
elements can lead to divergent replicas depending on the integration order. For
instance, let us consider two peers concurrently inserting $(Q,\,0)$ and
$(W,\,0)$. When they receive the operation from each other, the first peer
shifts the character $Q$ while the other shifts $W$. They ends up with $WQ$ and
$QW$ respectively. To solve this problem, the set $Z$ uses a function
$gen\beta$ to transform the mutable indices from $\mathcal{M}$ to immutable
indices from $\mathcal{I}$. Using the dense total order ($\mathcal{I},\,
<_\mathcal{I}$), we are able to retrieve the goal sequence. For instance, after
the concurrent insertions of $(Q,\,0)$ and $(W,\,0)$, the set $Z$ transforms
the shifting indices to $i_1$ and $i_2$ from $\mathcal{I}$. The execution order
of operations does not matter since the elements are identically ordered on
both replicas. The peers end up with either $QW$ or $WQ$. Trees are able to
represent the dense total order ($\mathcal{I},\, <_\mathcal{I}$). In this
regard, and taking into account the concurrency, the composition of $allocPath$
and $allocDis$ corresponds to $gen\beta$.

Finding an optimal function $gen\beta$ for any permutation $Y$ is
impossible. Indeed, there always exists an allocation function more suitable
for a particular case. Nonetheless, focusing on the random and the monotonic
editing behaviours provides a first answer restrained to text editing.

%%% Local Variables:
%%% mode: latex
%%% TeX-master: "../paper"
%%% End:
