
\section{Preliminaries}
\label{sec:preliminaries}

Optimistic replication is a replication strategy where each peer owns a local
copy (or replica) of the shared data. Since peers can directly modify this copy,
it improves both availability and responsiveness of the data. When a peer
performs an operation on the local data, the information is spread to all owners
of the shared data where they are integrated. Optimistic replication approaches
guarantee eventual consistency~\cite{bailis2013eventual},i.e., after the
receipt and integration of all operations by the members, the replicas converge
to an identical state. This section introduces the shared data structure that
guarantees eventual consistency and the dissemination protocol.

\subsection{Shared data structure for sequences}

To insert a character in a shared document, one cannot just use the standard
well-known operation \emph{insert element at index} of arrays and expect
guaranties on convergence. For instance, if two users concurrently edit a
sequence ERTY: the first user inserts a Q at index 0, while the second user
inserts a W at index 0. They respectively obtains QERTY and WERTY as resulting
sequence. On receipt of each other's operation they respectively obtain WQERTY
and QWERTY. Hence the need to inspect these concurrent cases to ensure
convergent replicas.

In this paper, we focus on a conflict-free replicated data type for sequences as
replicated data structure (\emph{sequence} for short). These sequences provide
two commutative operations: insert and delete. Indeed, these operations can be
integrated in any order as long as the deletion of an element is integrated
after its insertion.  The sequence is composed of elements and a unique and
immutable identifier is associated with each of these elements. The sequence can
be seen as a set of pairs linking an element to its identifier. A total order is
assumed on these identifiers and this order relation allows seeing the set of
pairs as a sequence. \TODO{The projection of a sequence on the elements builds the
document.}  When a collaborator performs an insert operation, it first allocates
the identifier of the element to insert. For instance, let us consider a
sequence $QWTY$ with the unique, immutable, and totally ordered integer
identifiers $1$, $2$, $4$, $8$ respectively. A collaborator inserts the element
$E$ between $W$ and $T$. The natural identifier that comes to mind is $3$. The
resulting sequence is $QWETY$. However, $R$ cannot be inserted between $E$ and
$T$ since $3$ and $4$ are contiguous. The space of identifiers must be enlarged
to handle the new insertion. If we consider identifiers as decimal numbers,
$3.1$ can be associated with the character $R$. If a new character has to be
inserted between $E$ and $R$, a new identifier will be allocated between $3$ and
$3.1$. Again, the space will be extended resulting in a new identifier $3.0$
suffixed by any non null integer. Let $X$ be the suffix, the order is preserved
since $(3 < 3.0.X < 3.1)$. Such growing identifiers are called variable-size
identifiers. The main objective is to keep the growth of the size of the
identifiers under acceptable boundaries.

%%\subsection{Variable-size Identifiers}

% Variable-size identifiers can be represented as a concatenation of basic
% elements (e.g. integers). The resulting sequence can be represented by a tree
% structure where the elements of the sequence are stored at the nodes and where
% the edges of the tree are labelled such that a path from the root to a node
% represents the identifier of the element stored at this node. For instance, the
% character $R$ in the previous example is accessible following the path composed
% of the edges labelled $3$ then $1$. More formally, a sequence is a tree
% $\mathcal{T}$ where each node contains a value i.e. an element of the sequence
% (over an alphabet $\mathcal{A}$). The tree $\mathcal{T}$ is a set of pairs
% $\langle\mathcal{P}\subset\{\mathbb{N}\}^*,\, \mathcal{A} \rangle$, i.e., each
% element has a path. Additionally, a total order
% ($\mathcal{P}$,$<_{\mathcal{P}}$) provides an ordering among the paths which
% allows to retrieve the order of the elements in the sequence.

% \noindent \textbf{Notation} A path composed of $p$ edges labelled
% $\ell_1,\ldots,\ell_p$ will be noted $[\ell_1.\ldots.\ell_p]$.

% \begin{figure}
%   \centering
%   \begin{tikzpicture}[scale=1.2]

\newcommand\Y{-30};
\newcommand\ADDY{-10};

  %% node to node
  \small
  \draw[dashed,thick] (0pt,0pt) -- node[anchor=south east]{0} (-70pt,-30pt);
  \draw[thick] (0pt,0pt) -- node[anchor=east]{1} (-50pt,-30pt); %% Q
  \draw[thick] (0pt,0pt) -- node[anchor=east]{2} (-30pt,-30pt); %% W
  \draw[thick] (0pt,0pt) -- node[anchor=east]{\textbf{3}} (-10pt,-30pt); %% E
  \draw[thick] (0pt,0pt) -- node[anchor=east]{4} ( 30pt,-30pt); %% T
  \draw[thick] (0pt,0pt) -- node[anchor=east]{8} ( 50pt,-30pt); %% Y
  \draw[dashed,thick] (0pt,0pt) -- node[anchor=south west]{9} ( 70pt,-30pt);

  \draw[thick] (-10pt,-30pt) -- node[anchor=west]{\textbf{1}} ( 10pt,-60pt); %% R
%  \draw[thick] (-10pt,-30pt) -- node[anchor=west]{\textbf{0}} (-10pt,-60pt);
%  \draw[thick] (-10pt,-60pt) -- node[anchor=east]{\textbf{X}} (  0pt,-90pt); %% ?


  %% node to element
  \draw[->] (-50pt,\Y pt) -- (-60pt, \ADDY +\Y pt);
  \draw[->] (-30pt,\Y pt) -- (-40pt, \ADDY +\Y pt);
  \draw[->] (-10pt,\Y pt) -- (-20pt, \ADDY +\Y pt);
  \draw[->] ( 30pt,\Y pt) -- ( 30pt, \ADDY +\Y pt);
  \draw[->] ( 50pt,\Y pt) -- ( 50pt, \ADDY +\Y pt);

  \draw[->] ( 10pt,2 * \Y pt) -- ( 10pt, \ADDY + 2 * \Y pt);
%  \draw[->] (  0pt,3 * \Y pt) -- (  0pt, \ADDY + 3 * \Y pt);

  %% nodes
  \draw[fill=black] ( 0pt,  0pt) circle (1pt); 
  \draw[fill=black] (-70pt, -30pt) circle (1pt); 
  \draw[fill=white] (-50pt, -30pt) circle (1pt); %% Q
  \draw[fill=white] (-30pt, -30pt) circle (1pt); %% W
  \draw[fill=white] (-10pt, -30pt) circle (1pt); %% E
  \draw[fill=white] ( 30pt, -30pt) circle (1pt); %% T
  \draw[fill=white] ( 50pt, -30pt) circle (1pt); %% Y
  \draw[fill=black] ( 70pt, -30pt) circle (1pt);

%  \draw[fill=white] (-10pt, 2 * \Y pt) circle (1pt); %% R
  \draw[fill=white] ( 10pt, 2 * \Y pt) circle (1pt); %% x
%  \draw[fill=white] (  0pt, 3 * \Y pt) circle (1pt); %% ?

  %% elements
  \draw[fill=white](-60pt,-4 + \ADDY + \Y pt)
  node{\textbf{Q}} +(-4pt,-4pt) rectangle +(4pt,4pt) ; %% Q
  \draw[fill=white](-40pt,-4 + \ADDY + \Y pt)
  node{\textbf{W}} +(-4pt,-4pt) rectangle +(4pt,4pt) ; %% W
  \draw[fill=white](-20pt,-4 + \ADDY + \Y pt)
  node{\textbf{E}} +(-4pt,-4pt) rectangle +(4pt,4pt) ; %% E
  \draw[fill=white]( 30pt,-4 + \ADDY + \Y pt)
  node{\textbf{T}} +(-4pt,-4pt) rectangle +(4pt,4pt) ; %% T
  \draw[fill=white]( 50pt,-4 + \ADDY + \Y pt)
  node{\textbf{Y}} +(-4pt,-4pt) rectangle +(4pt,4pt) ; %% Y

  \draw[fill=white]( 10pt,-4 + \ADDY + 2 * \Y pt)
  node{\textbf{R}} +(-4pt,-4pt) rectangle +(4pt,4pt) ; %% R
%  \draw[fill=white](  0pt,-4 + \ADDY + 3 * \Y pt)
%  node{\textbf{?}} +(-4pt,-4pt) rectangle +(4pt,4pt) ; %% ?
  \draw(0, -8+\ADDY+ 2.5*\Y pt);

\end{tikzpicture}

%   \caption{\label{fig:treemodelexample} Underlying 10-ary tree $\mathcal{T}$ of
%     a sequence. The paths $\mathcal{P}$ correspond to the concatenation of the
%     edges from the root to the elements. The elements are characters. Using the
%     total order ($\mathcal{P},\, <_\mathcal{P}$), the sequence of elements is
%     $QWERTY$.}
% \end{figure}

% Figure~\ref{fig:treemodelexample} shows the underlying 10-ary tree
% $\mathcal{T}$ representing a sequences. Like in the previous scenario, the
% initial sequence is $QWTY$ with the respective paths $[1]$, $[2]$, $[4]$ and
% $[8]$. The insertion of the character $E$ between the pairs
% $\langle [2],\, W\rangle$ and $\langle [4],\, T\rangle$ results in the
% following pair: $\langle [3],\, E \rangle$. Then the insertion of the character
% $R$ needs to start a new level since there is no room at the first level of the
% tree for a new path between $E$ and $T$. The resulting path may be $[3.1]$ if
% label one is chosen for the element $R$ at the second level. This would
% increase the depth of the tree in case there is an insertion between the
% elements $E$ and $R$. The new path would be $[3.0.X]$ where $0<X<10$ (recall
% that we assumed a 10-ary tree). The total order $(\mathcal{P},\,<_\mathcal{P})$
% allows retrieving the sequence $QWERTY$.

% %% \subsection{Disambiguation of concurrent cases}

% Two collaborators concurrently performing an operation on their respective
% replica may get different results after the integration of both
% operations. Indeed, $(\mathcal{P},\,<_\mathcal{P})$ is a total order when a
% single collaborator edits. However, it becomes a partial order when the editing
% involves several collaborators. Consequently, it is necessary to totally order
% the elements inserted by different collaborators. To this end, the
% disambiguation function $\delta$ associates a disambiguator to each pair of
% $\mathcal{T}$. $\delta: \mathcal{P}\times\mathcal{A} \rightarrow \mathcal{D}$
% such that ($\mathcal{D}$, $<_{\mathcal{D}}$) is a total order. The function
% $\delta$ is an accessor to additional values that are totally ordered even in
% presence of concurrency. Finally, the pairs in $\mathcal{T}$ can be totally
% ordered with a composition of the local total order ($\mathcal{P}$,
% $<_{\mathcal{P}}$) and the total order between collaborators ($\mathcal{D}$,
% $<_{\mathcal{D}}$). The composition leads to a total order ($\mathcal{T}$,
% $<_{\mathcal{T}}$).

% Figure~\ref{fig:desexample} depicts a tree containing 6 elements with only 5
% distinct paths (two values are associated with the path $[3]$). Similarly to
% the previous example, the initial sequence was $QWTY$, however, in this
% example, the two elements $E$ and $R$ are inserted between the pairs
% $\langle [2],\, W\rangle$ and $\langle [4],\,T\rangle$ by two different
% collaborators concurrently. For both elements, the resulting path is
% $[3]$. After the two elements are inserted, the sequence becomes either
% $QWERTY$ or $QWRETY$. Nevertheless, let
% $\delta(\langle [3],\, R \rangle) = \delta_R$ and
% $\delta(\langle [3],\, T\rangle) = \delta_T$. If we assume
% $\delta_R <_\mathcal{D} \delta_T$, the total order
% $(\mathcal{T}, <_\mathcal{T})$ gives the sequence $QWERTY$. It is worth noting
% that disambiguators are usually computed using a monotonically increasing
% variable and a unique collaborator identifier just like Lamport
% timestamps~\cite{lamport1978time}. Therefore, a collaborator cannot directly
% influence the final position of the character in the sequence using
% disambiguators. The sequence of the example could have ended in $QWRETY$ and it
% would have needed a correction.

% \begin{figure}
%   \centering
%   \begin{tikzpicture}[scale=1.2]

\newcommand\Y{-30};
\newcommand\ADDY{-10};

  %% node to node
  \small
  \draw[dashed,thick] (0pt,0pt) -- node[anchor=south east]{0} (-70pt,\Y pt);
  \draw[thick] (0pt,0pt) -- node[anchor=east]{1} (-50pt, \Y pt); %% Q
  \draw[thick] (0pt,0pt) -- node[anchor=east]{2} (-30pt, \Y pt); %% W
  \draw[thick] (0pt,0pt) -- node[anchor=east]{3} (  0pt, \Y pt); %% E R
  \draw[thick] (0pt,0pt) -- node[anchor=west]{4} ( 30pt, \Y pt); %% T
  \draw[thick] (0pt,0pt) -- node[anchor=west]{8} ( 50pt, \Y pt); %% Y
  \draw[dashed,thick] (0pt,0pt) -- node[anchor=south west]{9} ( 70pt, \Y pt);

  %% node to element
  \draw[->] (-50pt,\Y pt) -- (-50pt, \ADDY +\Y pt);
  \draw[->] (-30pt,\Y pt) -- (-30pt, \ADDY +\Y pt);
  \draw[->] (  0pt,\Y pt) -- (-10pt, \ADDY +\Y pt);
  \draw[->] (  0pt,\Y pt) -- ( 10pt, \ADDY +\Y pt);
  \draw[->] ( 30pt,\Y pt) -- ( 30pt, \ADDY +\Y pt);
  \draw[->] ( 50pt,\Y pt) -- ( 50pt, \ADDY +\Y pt);

  %% element to desambiguator
  \draw[<->,densely dotted](-50pt,-8+\ADDY + \Y pt)--(-50pt,2.75*\ADDY+\Y pt);
  \draw[<->,densely dotted](-30pt,-8+\ADDY + \Y pt)--(-30pt,2.75*\ADDY+\Y pt);
  \draw[<->,densely dotted](-10pt,-8+\ADDY + \Y pt)--(-10pt,2.75*\ADDY+\Y pt);
  \draw[<->,densely dotted]( 10pt,-8+\ADDY + \Y pt)--( 10pt,2.75*\ADDY+\Y pt);
  \draw[<->,densely dotted]( 30pt,-8+\ADDY + \Y pt)--( 30pt,2.75*\ADDY+\Y pt);
  \draw[<->,densely dotted]( 50pt,-8+\ADDY + \Y pt)--( 50pt,2.75*\ADDY+\Y pt);

  %% nodes
  \draw[fill=black] ( 0pt, 0 pt) circle (1pt); 
  \draw[fill=black] (-70pt, \Y pt) circle (1pt); 
  \draw[fill=white] (-50pt, \Y pt) circle (1pt); %% Q
  \draw[fill=white] (-30pt, \Y pt) circle (1pt); %% W
  \draw[fill=white] (  0pt, \Y pt) circle (1pt); %% E
  \draw[fill=white] ( 30pt, \Y pt) circle (1pt); %% T
  \draw[fill=white] ( 50pt, \Y pt) circle (1pt); %% Y
  \draw[fill=black] ( 70pt, \Y pt) circle (1pt);

  %% desambiguator
  \draw[fill=gray!20] (-50pt,-2.5+2.75*\ADDY+\Y pt)
  +(-2.5pt,-2.5pt) rectangle +(2.5pt,2.5pt);
  \draw[fill=gray!20] (-30pt,-2.5+2.75*\ADDY+\Y pt)
  +(-2.5pt,-2.5pt) rectangle +(2.5pt,2.5pt);
  \draw[fill=gray!20, thick] (-10pt,-2.5+2.75*\ADDY+\Y pt)
  +(-2.5pt,-2.5pt) rectangle +(2.5pt,2.5pt);
  \draw[fill=gray!20, thick] ( 10pt,-2.5+2.75*\ADDY+\Y pt)
  +(-2.5pt,-2.5pt) rectangle +(2.5pt,2.5pt);
  \draw[fill=gray!20] ( 30pt,-2.5+2.75*\ADDY+\Y pt)
  +(-2.5pt,-2.5pt) rectangle +(2.5pt,2.5pt);
  \draw[fill=gray!20] ( 50pt,-2.5+2.75*\ADDY+\Y pt)
  +(-2.5pt,-2.5pt) rectangle +(2.5pt,2.5pt);

  %% elements
  \draw[fill=white](-50pt,-4 + \ADDY + \Y pt)
  node{\textbf{Q}} +(-4pt,-4pt) rectangle +(4pt,4pt) ; %% Q
  \draw[fill=white](-30pt,-4 + \ADDY + \Y pt)
  node{\textbf{W}} +(-4pt,-4pt) rectangle +(4pt,4pt) ; %% W
  \draw[fill=white](- 10pt,-4 + \ADDY + \Y pt)
  node{\textbf{E}} +(-4pt,-4pt) rectangle +(4pt,4pt) ; %% E
  \draw[fill=white]( 10pt,-4 + \ADDY + \Y pt)
  node{\textbf{R}} +(-4pt,-4pt) rectangle +(4pt,4pt) ; %% R
  \draw[fill=white]( 30pt,-4 + \ADDY + \Y pt)
  node{\textbf{T}} +(-4pt,-4pt) rectangle +(4pt,4pt) ; %% T
  \draw[fill=white]( 50pt,-4 + \ADDY + \Y pt)
  node{\textbf{Y}} +(-4pt,-4pt) rectangle +(4pt,4pt) ; %% Y

  
  \begin{scope}[shift={(20pt, 2.5*\Y pt)}]    
    \draw[fill=white](0,0)node{\textbf{E}} +(-4pt, -4pt) rectangle +(4pt, 4pt);
    \scriptsize
    \draw (4pt, 0)node[anchor=west]{element};
    \draw[fill=gray!20] (0,\ADDY pt) +(-2.5pt, -2.5pt)rectangle+(2.5pt, 2.5pt);
    \draw (3pt, \ADDY pt) node[anchor=west]{disambiguator};
  \end{scope}
  %% spacing
  \draw(0, -8+\ADDY+ 3*\Y pt);

\end{tikzpicture}

%   \caption{\label{fig:desexample}The 10-ary tree of a sequence including
%     concurrent insertions. Using only ($\mathcal{P},\, <_\mathcal{P}$) could
%     lead to either $QWERTY$ or $QWRETY$. The disambiguators forces the final
%     result to converge to $QWERTY$.}
% \end{figure}

% %% \subsection{Choosing the rightful path}
% \label{sec:choosing}

The most critical part of such sequences consists in creating the
paths. Algorithm~\ref{algo:crdtabstract} shows the general outlines of these
sequences. It divides the operations into two different parts, the local and the
remote phases of the optimistic replication paradigm. As we can see, the local
part of the insert operation concentrates most of the complexity where the
algorithm has to generate the identifier composed of the path, i.e., the list of
integers (cf. Line~\ref{line:allocpath}) and a disambiguator to ensure a global
total order (cf. Line~\ref{line:allocdes}).

\begin{algorithm}
  
\small
\algrenewcommand{\algorithmiccomment}[1]{\hskip2em$\rhd$ #1}

\newcommand{\comment}[1]{$\rhd$ #1}


\algblockdefx[initially]{initially}{endInitially}
  [0] {\textbf{INITIALLY:}} 

\algblockdefx[local]{local}{endLocal}
  [0] {\textbf{LOCAL UPDATE:}}

\algblockdefx[received]{received}{endReceived}
  [0] {\textbf{RECEIVED UPDATE:}}

\algblockdefx[onInsert]{onLocal}{endOnLocal}
  [0] {\textbf{on} insert ($p \in \mathcal{I},\,\alpha \in \mathcal{A},\,
   q\in\mathcal{I}$):}
  [0] {\textbf{on} delete ($i \in \mathcal{I}$):} 

\algblockdefx[onRemote]{onRemote}{endOnRemote}
  [0] {\textbf{on} insert ($i\in\mathcal{I}$):\hfill\comment{once per 
  distinct triple in $\mathcal{I}$}}
  [0] {\textbf{on} delete ($i\in\mathcal{I}$):\hfill\comment{after the 
  remote $insert(i)$ is done}} 

\newcommand{\LINEFOR}[2]{%
  \algorithmicfor\ {#1}\ \algorithmicdo\ {#2} %
  }

\newcommand{\LINEIFTHEN}[2]{%
  \algorithmicif\ {#1}\ \algorithmicthen\ {#2} %
  }

\newcommand{\INDSTATE}[1][1]{\State\hspace{\algorithmicindent}}

\begin{algorithmic}[1]
  \Statex
  \initially
    \State $\mathcal{T} \leftarrow \varnothing$; \hfill \comment{structure of
     the CRDT for sequences}
  \endInitially
  
  \local
    \onLocal
    \State \textbf{let} $path \leftarrow allocPath(p.P,\,q.P)$; \label{line:allocpath}
    \State \textbf{let} $dis \leftarrow allocDis(p,\, path,\, q)$; \label{line:allocdes}
    \State $broadcast('insert',\, \langle path,\, \alpha,\, dis \rangle)$;
    \endOnLocal
    \INDSTATE $broadcast('delete',\,i)$;
  \endLocal
  
  \received
    \onRemote
    \State $\mathcal{T} \leftarrow \mathcal{T} \cup i$;
    \endOnRemote
    \INDSTATE $\mathcal{T} \leftarrow \mathcal{T}\, \backslash\, i$; 
  \endReceived
  
\end{algorithmic}

  \caption{\label{algo:crdtabstract}General outlines of a sequence with
    variable-size identifiers.}
\end{algorithm}

\TODO{Let $\mathcal{I}$ be the set of unique triples
$\mathcal{I}: \mathcal{P}\times\mathcal{A}\times\mathcal{D}$. The set of the
elements of a sequence is a subset of $\mathcal{I}$. For any element $i$ of a
sequence ($i \in \mathcal{I}$), let $i.P$, $i.A$, $i.D$ be the respective
accessors to the path, the element, and the disambiguator of element $i$. The
function $allocPath$ chooses a path in the tree between two other paths $p.P$
and $q.P$ where $p<_{\mathcal{T}}q$. This means that element $i$ has to be
inserted between two elements $p$ and $q$ such that $p$ precedes $q$ in the
sequence. However, since it is always create new levels in the tree and thus
new sub-trees, the number of possible paths is infinite, and so is the number
of $allocPath$ functions. Nevertheless, the function $allocPath$ should choose
among the all the possible paths the one having the smallest length in order to
have smaller identifiers keeping good performance. This observation reduces
considerably the number of possible $allocPath$ functions. Still, the
allocation of paths without an \emph{a priori} knowledge of the final sequence
is a non-trivial problem (\ref{sec:problem} depicts the problem abstracted from
the collaborative editing context).}

% \begin{figure}
%   \centering
%   \begin{tikzpicture}[scale=1]

  %% node to node
  \small
  \draw[dashed, thick] (0pt,0pt) -- node[anchor=south east]{0} (-70pt,-40pt);
  \draw[thick] (0pt,0pt) -- node[anchor=east]{1} (-50pt,-40pt);
  \draw[thick] (0pt,0pt) -- node[anchor=east]{2} (-30pt,-40pt);
  \draw[thick] (0pt,0pt) -- node[anchor=east]{3} (-10pt,-40pt);
  \draw[thick] (0pt,0pt) -- node[anchor=west]{4} ( 10pt,-40pt);
  \draw[thick] (0pt,0pt) -- node[anchor=west]{5} ( 30pt,-40pt);
  \draw[thick] (0pt,0pt) -- node[anchor=west]{6} ( 50pt,-40pt);
  \draw[dashed, thick] (0pt,0pt) -- node[anchor=south west]{9} ( 70pt,-40pt);

  %% node to element
  \draw[->] (-50pt,-40pt) -- (-50pt,-50pt);
  \draw[->] (-30pt,-40pt) -- (-30pt,-50pt);
  \draw[->] (-10pt,-40pt) -- (-10pt,-50pt);
  \draw[->] ( 10pt,-40pt) -- ( 10pt,-50pt);
  \draw[->] ( 30pt,-40pt) -- ( 30pt,-50pt);
  \draw[->] ( 50pt,-40pt) -- ( 50pt,-50pt);

  %% element to desambiguator
  \draw[->,densely dashdotted] ( -50pt,-58pt) -- ( -50pt,-68.5pt);
  \draw[->,densely dashdotted] ( -30pt,-58pt) -- ( -30pt,-68.5pt);
  \draw[->,densely dashdotted] ( -10pt,-58pt) -- ( -10pt,-68.5pt);
  \draw[->,densely dashdotted] (  10pt,-58pt) -- (  10pt,-68.5pt);
  \draw[->,densely dashdotted] (  30pt,-58pt) -- (  30pt,-68.5pt);
  \draw[->,densely dashdotted] (  50pt,-58pt) -- (  50pt,-68.5pt);

  \draw[fill=black] (  0pt,  0pt) circle (1pt);
  \draw[fill=black] (-70pt,-40pt) circle (1pt);
  \draw[fill=white] (-50pt,-40pt) circle (1pt);
  \draw[fill=white] (-30pt,-40pt) circle (1pt);
  \draw[fill=white] (-10pt,-40pt) circle (1pt);
  \draw[fill=white] ( 10pt,-40pt) circle (1pt);
  \draw[fill=white] ( 30pt,-40pt) circle (1pt);
  \draw[fill=white] ( 50pt,-40pt) circle (1pt);
  \draw[fill=black] ( 70pt,-40pt) circle (1pt);

  %% elements
  \draw[fill=white](-50pt,-54pt)
  node{\textbf{Q}}+(-4pt,-4pt)rectangle+(4pt,4pt) ;
  \draw[fill=white](50pt,-54pt)
  node{\textbf{Y}} +(-4pt,-4pt) rectangle +(4pt,4pt) ;
  \draw[fill=white]( 10pt,-54pt)
  node{\textbf{R}} +(-4pt,-4pt) rectangle +(4pt,4pt) ;
  \draw[fill=white] ( -30pt,-54pt)
  node{\textbf{W}} +(-4pt,-4pt) rectangle +(4pt,4pt) ;
  \draw[fill=white] ( -10pt,-54pt)
  node{\textbf{E}} +(-4pt,-4pt) rectangle +(4pt,4pt) ;
  \draw[fill=white]( 30pt,-54pt)
  node{\textbf{T}} +(-4pt,-4pt) rectangle +(4pt,4pt) ;

  %% desambiguator
  \draw[fill=gray!20] (-50pt,-71pt) +(-2.5pt,-2.5pt) rectangle +(2.5pt,2.5pt);
  \draw[fill=gray!20] (-30pt,-71pt) +(-2.5pt,-2.5pt) rectangle +(2.5pt,2.5pt);
  \draw[fill=gray!20] (-10pt,-71pt) +(-2.5pt,-2.5pt) rectangle +(2.5pt,2.5pt);
  \draw[fill=gray!20] ( 10pt,-71pt) +(-2.5pt,-2.5pt) rectangle +(2.5pt,2.5pt);
  \draw[fill=gray!20] ( 30pt,-71pt) +(-2.5pt,-2.5pt) rectangle +(2.5pt,2.5pt);
  \draw[fill=gray!20] ( 50pt,-71pt) +(-2.5pt,-2.5pt) rectangle +(2.5pt,2.5pt);

  %% insertion order
  \draw[->,dashed] (-50pt, -90pt) -- node[anchor=north]{insertion order}
  (50pt, -90pt);

%%%%%%%%%%%%%%%%%%%%%%%%%%%%%%%%%%%%%%%%%%%%%%%%%%%%%%%%%%%%%%%%%%%%%%

  \draw[dashed] (80pt,0pt) -- (80pt,-132pt);
  
  \draw (-70pt, -132pt)node[anchor=west]{\textbf{Nearly optimal}};


\begin{scope}[shift={(135pt,0pt)}]

\newcommand\Y{-19}
\newcommand\ADDY{-8}

  \draw (45pt, -132pt) node[anchor=east]{\textbf{Worst}};  

  %% node to node
  \small
  \draw[thick] (0pt,0pt) -- node[anchor=south east]{0} (-40pt,\Y pt);
  \draw[thick] (0pt,0pt) -- node[anchor=east]{1} (30pt, \Y pt); %% Y
  \draw[thick] (-40pt, \Y pt) -- node[anchor=north]{1} (15pt, 2 * \Y pt); %% T
  \draw[thick] (-40pt, \Y pt) -- node[anchor=east]{0} (-40pt, 2 * \Y pt); %% 0
  \draw[thick] (-40pt, 2*\Y pt) -- node[anchor=north]{1} (0pt, 3 * \Y pt); %% R
  \draw[thick] (-40pt, 2*\Y pt)-- node[anchor=east]{0}(-40pt, 3 * \Y pt); %% 0
  \draw[thick] (-40pt, 3*\Y pt) -- node[anchor=north]{1}(-15pt,4 * \Y pt); %% E
  \draw[thick] (-40pt, 3*\Y pt) -- node[anchor=east]{0}(-40pt,4 * \Y pt); %% 0
  \draw[thick] (-40pt, 4*\Y pt) -- node[anchor=north]{1}(-25pt,5 * \Y pt); %% W
  \draw[thick] (-40pt, 4*\Y pt) -- node[anchor=east]{0}(-40pt,5 * \Y pt); %% 0
  \draw[thick] (-40pt, 5*\Y pt) -- node[anchor=east]{1}(-35pt,6 * \Y pt); %% Q

  \draw[dashed, thick] (0pt,0pt) -- node[anchor=south west]{9} (40pt,\Y pt);

  %% node to element
  \draw[->] ( 30pt, \Y pt) -- ( 30pt, \ADDY + \Y pt); %% Y
  \draw[->] ( 15pt, 2* \Y pt) -- ( 15pt, \ADDY + 2 *\Y pt); %% T
  \draw[->] (  0pt, 3 *\Y pt) -- (  0pt, \ADDY + 3 *\Y pt); %% R
  \draw[->] (-15pt, 4 *\Y pt) -- ( -15pt, \ADDY + 4 *\Y pt); %% E
  \draw[->] (-25pt, 5 *\Y pt) -- ( -25pt, \ADDY + 5 *\Y pt); %% W
  \draw[->] (-35pt, 6 *\Y pt) -- ( -35pt, \ADDY + 6 *\Y pt); %% Q

  %% element to desambiguator
  \draw[->,densely dashdotted]
  ( 30pt, \ADDY + \Y pt) -- ( 30pt,2.75*\ADDY+\Y pt); %% Y
  \draw[->,densely dashdotted]
  ( 15pt, \ADDY + 2* \Y pt) -- ( 15pt,2.75*\ADDY+ 2* \Y pt); %% T
  \draw[->,densely dashdotted]
  ( 0pt, \ADDY + 3* \Y pt) -- (  0pt,2.75*\ADDY+ 3* \Y pt); %% R
  \draw[->,densely dashdotted]
  ( -15pt, \ADDY + 4 *\Y pt) -- ( -15pt,2.75*\ADDY+ 4* \Y pt); %% E
  \draw[->,densely dashdotted]
  ( -25pt, \ADDY + 5 *\Y pt) -- ( -25pt,2.75*\ADDY+ 5*\Y pt); %% W
  \draw[->,densely dashdotted]
  ( -35pt, \ADDY + 6* \Y pt) -- ( -35pt,2.75*\ADDY+ 6*\Y pt); %% Q

  %% node
  \draw[fill=black] (0pt,0pt) circle (1pt); %% rooot
  \draw[fill=white] ( 30pt, \Y pt) circle (1pt); %% Y
  \draw[fill=white] (-40pt, \Y pt) circle (1pt); %% 0
  \draw[fill=white] ( 15 pt, 2 * \Y pt) circle (1pt); %% T
  \draw[fill=white] (-40pt, 2 * \Y pt) circle (1pt); %% 0
  \draw[fill=white] (  0 pt, 3 * \Y pt) circle (1pt); %% R
  \draw[fill=white] (-40pt, 3 * \Y pt) circle (1pt); %% 0
  \draw[fill=white] (-15 pt, 4 * \Y pt) circle (1pt); %% E
  \draw[fill=white] (-40pt, 4 * \Y pt) circle (1pt); %% 0
  \draw[fill=white] (-25 pt, 5 * \Y pt) circle (1pt); %% W
  \draw[fill=white] (-40pt, 5 * \Y pt) circle (1pt); %% 0
  \draw[fill=white] (-35 pt, 6 * \Y pt) circle (1pt); %% Q

  \draw[fill=black] ( 40pt, \Y pt) circle (1pt);


  %% elements
  \draw[fill=white] ( 30pt, -4 + \ADDY + \Y pt)
  node{\textbf{Y}} +(-4pt,-4pt) rectangle +(4pt,4pt) ; %% Y
  \draw[fill=white] ( 15pt, -4 + \ADDY +  2 *\Y pt)
  node{\textbf{T}} +(-4pt,-4pt) rectangle +(4pt,4pt) ; %% T
  \draw[fill=white] (  0pt, -4 + \ADDY +  3* \Y pt)
  node{\textbf{R}} +(-4pt,-4pt) rectangle +(4pt,4pt) ; %% R
  \draw[fill=white] (-15pt, -4 + \ADDY + 4 *\Y pt)
  node{\textbf{E}} +(-4pt,-4pt) rectangle +(4pt,4pt) ; %% E
  \draw[fill=white] (-25pt, -4 + \ADDY + 5 * \Y pt)
  node{\textbf{W}} +(-4pt,-4pt) rectangle +(4pt,4pt) ; %% W
  \draw[fill=white] (-35pt, -4 + \ADDY + 6 *\Y pt)
  node{\textbf{Q}} +(-4pt,-4pt) rectangle +(4pt,4pt) ; %% Q

  %% desambiguator
  \draw[fill=gray!20]( 30pt, -2.5 + 2.75 * \ADDY + \Y pt)
  +(-2.5pt,-2.5pt) rectangle +(2.5pt,2.5pt);
  \draw[fill=gray!20]( 15pt, -2.5 + 2.75 * \ADDY +2 *\Y pt)
  +(-2.5pt,-2.5pt) rectangle +(2.5pt,2.5pt);
  \draw[fill=gray!20](  0pt, -2.5 + 2.75 * \ADDY + 3*\Y pt)
  +(-2.5pt,-2.5pt) rectangle +(2.5pt,2.5pt);
  \draw[fill=gray!20](-15pt, -2.5 + 2.75 * \ADDY +4*\Y pt )
  +(-2.5pt,-2.5pt) rectangle +(2.5pt,2.5pt);
  \draw[fill=gray!20](-25pt, -2.5 + 2.75 * \ADDY + 5*\Y pt)
  +(-2.5pt,-2.5pt) rectangle +(2.5pt,2.5pt);
  \draw[fill=gray!20](-35pt, -2.5 + 2.75 * \ADDY +6*\Y pt) 
  +(-2.5pt,-2.5pt) rectangle +(2.5pt,2.5pt);

  %% insertion order
  \draw[->,dashed] (30pt, 3 * \Y pt) -- node[anchor=west,align=left]
  {\ \ insertion\\ order} (-25pt, 7.5 * \Y pt);


\end{scope}

\end{tikzpicture}

%   \caption{\label{fig:allocpathexample} Two trees filled with the resulting
%     identifiers of two different permutations resulting in an identical
%     sequence $QWERTY$. They use the same function $allocPath$ which allocates
%     the leftmost branch in the tree. All paths of the nearly optimal case have
%     a length of 1 while the tree of the worst case grows up to a depth of 6.}
% \end{figure}

% Figure~\ref{fig:allocpathexample} illustrates the difficulties of designing a
% function to allocate the paths. It represents the underlying trees of two
% sequences using the allocation function: they allocate the leftmost branch
% available at the lowest depth possible. In both cases the final sequence is
% $QWERTY$, however, the letters are not inserted in the same order. In the first
% case, $Q$ is inserted first at position 0 (initially the sequence is empty),
% followed by $W$ at position 1 (after $Q$) then $E$ is inserted at position 3
% (after $W$), etc. We call the sequence of insert operations $[(Q,\,0)$,
% $(W,\,1)$, $(E,\,2)$, $\ldots]$ the \emph{editing sequence}. In the second
% case, the letter $Y$ is inserted first at position 0 as the sequence is
% initially empty. Then $T$ is inserted. However as the final intended word is
% $QWERTY$, $T$ has to be inserted at a position before $Y$ that represents the
% current state of the sequence. $T$ is thus inserted at position 0 shifting $Y$
% to position $1$, etc. The editing sequence that corresponds to this case
% is thus $[(Y,\,0)$, $(T,\,0)$, $(R,\,0)$, $\ldots]$.

% \begin{itemize}[leftmargin=*]
% \item Case 1: Since the function $allocPath$ allocates the leftmost branches,
%   the following editing sequence $[(Q,\,0)$, $(W,\,1)$, $(E,\,2)$, $\ldots]$
%   leads to the paths $\langle [1],\, Q\rangle$, $\langle [2],\, W\rangle$,
%   $\langle [3],\, E\rangle$, etc. In this case, the depth of the tree never
%   grows. In this regard, this function $allocPath$ is very efficient in terms
%   of the size of the allocated identifiers.

% \item Case 2: The editing sequence $[(Y,\,0)$, $(T,\,0)$, $(R,\,0)$, $\ldots]$
%   leads to an increase of the depth of the tree at each element
%   insertion. Indeed, as an element gets the smallest value at its level
%   (cf. the allocation function), there is no way to insert a new element before
%   at the same level hence the new level. The resulting sequence is
%   $\langle [1],\, Y\rangle$, $\langle [0.1],\, T\rangle$,
%   $\langle [0.0.1],\, R \rangle$, etc. Consequently, the size of the paths
%   grows very fast.
% \end{itemize}

% This example shows how the insertion order impacts the length of the allocated
% paths. Unfortunately, the insertion order cannot be predicted, nor the size of
% the final sequence. Prior work on sequences often made the assumption of a
% left-to-right editing due to observations made on
% corpus~\cite{preguica2009commutative, weiss2009logoot}. However, there exist
% human edited documents that do not correspond to this kind of
% editing~\cite{nedelec2013lseq}. Indeed, the editing depends on the type of the
% document and to the activity for example when correcting a document the editing
% in mainly random as the insertions/deletions may correspond to
% errors. Consequently, we are looking for an allocation function which provides
% identifiers with a sub-linear spatial complexity compared to the number of
% insertions whatever is the editing sequence.

\subsection{Scalable information dissemination}

To provide the eventual consistency of the aforementioned optimistic
replication approach, the application must disseminate the result of operations
to all participants of the editing session, i.e., broadcast the
operations. Since the number of these participants is possibly large,
participants cannot afford the full knowledge of the membership. Instead, the
common scalable approach consists in keeping a small list of known members
(referred as \emph{neighbors}) and send them operations. Then, each neighbors
forwards the operations to their own neighborhood. Thus, from
neighbors-to-neighbors, the operations reach all members.

The recent WebRTC technology allows establishing peer-to-peer connections inside
web browsers. As such, it allows creating an overlay network between web
browsers. Nevertheless, WebRTC imposes a complex and costly connection protocol.
Indeed, as depicted in \TODO{Figure}, a user $u_1$ wishing to establish a
connection with a user $u_2$ must create an offer ticket and send it to $u_2$
through a mediator (e.g. by mail, dedicated servers, peers already connected in
the network). Then, $u_2$ must answer with another offer (refereed as
\emph{stamped ticket}) using a mediator too. Finally, $u_1$ must confirm the
connection to establish the bidirectional link. From its complexity, WebRTC
encourages to make as small connections as possible, especially since the range
of users includes small devices (e.g. smartphones, tablets, raspberry pi) with
limited capacities.

%%% Local Variables: 
%%% mode: latex
%%% TeX-master: "../paper"
%%% End: 
