
\section{Message delivery protocol}
\label{sec:messagedelivery}
This section provides the additional algorithms of the messaging between
peers. The distributed collaborative editor \EDITORNAME{} uses these protocols
to ensure that operations are eventually integrated, only once, and the
deletion of an element is always integrated after its insertion.

\begin{algorithm}[h]
  
\small
\algrenewcommand{\algorithmiccomment}[1]{\hskip2em$\rhd$ #1}

\algblockdefx[initially]{initially}{endInitially}
  [0] {\textbf{INITIALLY:}} 

\algblockdefx[local]{local}{endLocal}
  [0] {\textbf{LOCAL UPDATE:}}

\algblockdefx[received]{received}{endReceived}
  [0] {\textbf{RECEIVED UPDATE:}}

\algblockdefx[onInsert]{onLocal}{endOnLocal}
  [0] {\textbf{on} insert (args):}
  [0] {\textbf{on} delete (args):} 

\algblockdefx[onRemote]{onRemote}{endOnRemote}
  [0] {\textbf{on} insert (args):}
  [0] {\textbf{on} delete (args):} 

\newcommand{\LINEFOR}[2]{%
  \algorithmicfor\ {#1}\ \algorithmicdo\ {#2} %
  }

\newcommand{\LINEIFTHEN}[2]{%
  \algorithmicif\ {#1}\ \algorithmicthen\ {#2} %
  }

\newcommand{\INDSTATE}[1][1]{\State\hspace{\algorithmicindent}}

\begin{algorithmic}[1]
  \State \textbf{let} $r$ the old ivv on peer $p_i$
  \State \textbf{let} $s$ the new ivv on peer $p_i$
  \State \textbf{let} $m$ the incoming message from peer $p_j$
  \State \textbf{let} $\langle type,\, args \rangle$ the specification of $m$

  \Statex
  \initially
    \State \LINEFOR{$k$ \textbf{from} $0$ \textbf{to} $|r| - 1$}
    {$s[k] \leftarrow \varnothing$;}
  \endInitially
  
  \local
    \State \LINEFOR{$k$ \textbf{from} $0$ \textbf{to} $|r|- 1$}
    {$s[k] \leftarrow r[k]$;}
    \onLocal
    \State $s[i] \leftarrow r[i].add(r[i].last() + 1)$;
    \State $broadcast('insert',\,alloc(args))$;
    \endOnLocal
    \INDSTATE $broadcast('delete',\,args)$;
  \endLocal
  
  \received
    \State \LINEFOR{$k$ \textbf{from} $0$ \textbf{to} $|r|-1$}
    {$s[k]\leftarrow r[k]$;}
    \onRemote
    \State \textbf{let} $\langle src,\,cnt \rangle \leftarrow unique(args)$;
    \If  {\label{line:unique}$(\neg(r[src].contains(cnt)))$}
      \State $s[src] \leftarrow r[src].add(cnt)$;
      \State $deliver(m)$;
    \EndIf
    \endOnRemote
    \INDSTATE \textbf{let} $\langle src,\,cnt \rangle \leftarrow unique(args)$;
    \INDSTATE \textbf{if} $(r[src].contains(cnt))))$
      \INDSTATE \hspace{\algorithmicindent}\textbf{then} $deliver(m)$;
      \INDSTATE \hspace{\algorithmicindent}\textbf{else} $delay(m)$;
      \label{line:delay}
  
  \endReceived
  
\end{algorithmic}

  \caption{\label{algo:delivery}Delivery protocol}
\end{algorithm}

Algorithm~\ref{algo:delivery} describes the delivery protocol running at each
peer. It highlights the optimistic replication scheme of CRDTs by dividing the
operations between the local update and the received update.  Since different
implementations are possible, the functions relative to the interval structure
are left aside. Semantically, the $r[i].add$ function adds the value in
parameters to the vector of intervals. The $r[i].last$ function gets the
highest value of the vector of intervals. The function $alloc$ is an
aggregation of $allocPath$ and $allocDis$. The function $unique$ extracts the
unique site identifier and counter from a triple.

Line~\ref{line:unique} prevents the application from multiple
receptions. Without any causality tracking mechanism, a CRDT for sequences is
not able to provide the idempotency property of CRDTs. Without idempotency, the
replicas may diverge. Indeed, the remote part of the delete operations removes
information from the data structure. When the causality tracking implicitly
keeps the summary of all operations, it implies that a non-existing element
cannot be mistaken for an element inserted then deleted.

\begin{asparadesc}
\item[Example \EXAMPLE{}:] The following example depicts the need to integrate
  the operation only once in presence of potential duplication of messages. A
  first peer generates two operations with the unique element $e$ ($insert(e)
  \rightarrow delete(e)$). When the $insert(e)$ operation arrives to a second
  peer, it is delivered. Similarly, when the $delete(e)$ operation arrives and
  sees the element $e$ in the replicated structure, it is delivered, leading to
  the removal of the element $e$. However, somehow, a copy of the $insert$
  message arrives to the second peer. Since the element $e$ does not exist in
  the structure any more, this peer will assume that it receives this operation
  for the first time and will apply it. Since the first peer does not
  necessarily receive the copy of the $insert$ message, or receives the copy
  before performing the $delete(e)$ operation, the replicas of the two peers
  may become divergent.
\end{asparadesc}


As stated in Section~\ref{sec:preliminaries}, CRDTs require a reliable
broadcast to provide strong eventual consistency. Nevertheless, \EDITORNAME{}
uses a gossip dissemination protocol which scales in number of peers, yet
unreliable.

\begin{algorithm}[h]
  
\small

\algrenewcommand{\algorithmiccomment}[1]{\hskip2em$\rhd$ #1}

\algblockdefx[received]{received}{endReceived}
  [0] {\textbf{RECEIVED UPDATE:}}

\algblockdefx[onRemoteBis]{onRemoteBis}{endOnRemoteBis}
  [0] {\textbf{on} antiEntropy (args):}
  [0] {}

\newcommand{\LINEFOR}[2]{%
  \algorithmicfor\ {#1}\ \algorithmicdo\ {#2} %
  }

\newcommand{\LINEIFTHEN}[2]{%
  \algorithmicif\ {#1}\ \algorithmicthen\ {#2} %
  }

\newcommand{\INDSTATE}[1][1]{\State\hspace{\algorithmicindent}}

\begin{algorithmic}[1]
  \received
  \State \textbf{on} antiEntropy ($args$):
  \INDSTATE \textbf{let} $ivv \leftarrow args$;
  \INDSTATE \textbf{let} $deltas \leftarrow diff(r,\,ivv)$;
  \INDSTATE \LINEFOR {($\delta$ \textbf{in} $deltas$)}
            {$sendTo(p_j,\, retrieve(\delta))$;}
  \endReceived
  
\end{algorithmic}

  \caption{\label{algo:antientropy}Anti-entropy protocol}
\end{algorithm}

Algorithm~\ref{algo:antientropy} depicts the anti-entropy event added to the
main delivery algorithm. This algorithm guarantees that all the operations will
be eventually delivered to all peers.  A peer $p_j$ unevenly sends a message
containing its interval version vector to another peer $p_i$ triggering the
event $antiEntropy$. The function $diff$ calculates the differences between the
interval version vectors (from $p_i$ and $p_j$). The function $retrieve$
searches for each spotted difference and sends it back to $p_j$. Being costly,
this protocol starts rarely.
