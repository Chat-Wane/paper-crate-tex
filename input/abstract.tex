\begin{abstract}
  Distributed real-time editors such as Google Docs, or Etherpad allow people to
  work distributed in space and across organizations. However, current solutions
  either lack of scalability or suffer of issues in privacy, censorship,
  single-point-of-failure, ease of deployement, economic intelligence,
  restrictions in terms of service. Building a fully decentralized, scalable,
  easy-to-deploy real-time editor requires to find a difficult compromise
  between time, space, and communication complexities with respect to the number
  of users and the size of shared documents. In this paper, we take advantage of
  WebRTC to build \CRATE; an easy-to-deploy real-time editor that runs on a
  serverless network of browsers. We propose a new class of complexity in order
  to support large number of users working on large documents. We extensively
  describe each core components, we provide the proof of the polylarithmic
  upper-bound on the space complexity of the shared sequence. We evaluate \CRATE
  in large-scale experiments on the Grid'5000 testbed until 600 participants. We
  observed logarithmic progession of generated traffic according the size of
  shared document and the number of participants.
\end{abstract}

%%% Local Variables: 
%%% mode: latex
%%% TeX-master: "../paper"
%%% End: 
