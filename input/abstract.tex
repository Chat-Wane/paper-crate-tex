\begin{abstract}
  Distributed real-time editors such as Google Docs, or Etherpad allow
  people to work distributed in space and across
  organizations. However, main stream solutions rely on centrals
  servers with issues in privacy, censorship, single-point-of-failure,
  economic intelligence and scalability. 
%
Building a fully
  decentralized, scalable real-time editor requires to find a
  difficult compromise between time, space, and communication
  complexities with respect to the number of users and the size of
  shared documents.
%
  In this paper, we propose \CRATE; n real-time editor that runs on a
  network of browsers i.e. without servers. \CRATE relies on a new
  shared document structure in order to support large number of users
  working on large documents. Thanks to WebRTC, \CRATE can be easily
  deployed.
%
We extensively
  describe each core component and provide the proof of the
  polylogarithmic upper-bound on the space complexity of the shared
  document. We evaluate \CRATE in large-scale experiments on the
  Grid'5000 testbed uptil 600 participants. As expected, we observe
  a logarithmic progession of the generated traffic according to the
  size of the
  shared document and the number of participants.
\end{abstract}

%%% Local Variables: 
%%% mode: latex
%%% TeX-master: "../paper"
%%% End: 
