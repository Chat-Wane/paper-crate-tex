
\begin{abstract}

  The recent development of distributed collaborative editors has renewed the
  interest for efficient data structures that can allow a huge number of
  simultaneous users. Indeed, existing editors lack scalability in terms of the
  number of users, concurrency, and the size of the produced
  documents. Additionally, centralised editors suffer from privacy issues,
  single-point-of-failure, economic intelligence issues, and restrictions in
  terms of service.  Several replicated data structures implementing sequences
  have been proposed to alleviate most of the aforementioned issues. Yet, they
  have an unsatisfying space complexity. Either they use tombstones and then
  deleted elements are only hidden to the user or they allocate to each element
  of the sequence an identifier that may be as large as the number of elements
  in the sequence. This paper proposes an identifier allocation function named
  \NAME{}. It provides identifiers enjoying a sub-linear space complexity
  without being application specific and proves the polylogarithmic size of
  these identifiers. Moreover, this paper provides all the outlines to develop
  distributed collaborative editors that scale in all the aforementioned
  dimensions. Using these outlines, we developed a prototype called
  \EDITORNAME{} ({\sc c}ollabo{\sc rat}ive {\sc e}ditor) that we evaluated on
  the Grid'5000 testbed. The results show that such editors, being
  decentralised and scalable, could constitute a serious competitor to current
  trending editors. In particular, it allows massive collaborative authoring of
  huge documents which opens the field to new kinds of distributed
  applications.
\end{abstract}

%%% Local Variables: 
%%% mode: latex
%%% TeX-master: "../paper"
%%% End: 
