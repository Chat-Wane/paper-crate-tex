\begin{abstract}
  Distributed real-time editors such as Google Docs, or Etherpad allow
  users distributed in space and organizations to collaborate easily
  with just web browsers. However, main stream editors rely on central
  servers with issues in privacy, single point of failure and
  scalability.  Decentralized editors can tackle privacy problems but
  scalability issues remain. According to the different approaches,
  decentralized editors deliver poor performances on cpu, space or
  communications in presence of large number of users or large shared
  documents. 
% 
  In this paper, we propose \CRATE; a decentralized real-time editor
  that supports large number of users working on large
  documents. Compared to state of art, we prove a polylogarithmic
  upper-bound on the space complexity of the shared document without
  changing degrading time and communication complexities. Thanks to
  WebRTC, \CRATE can be easily deployed on a network of browsers
  i.e. without servers.
%4
  We evaluate \CRATE in large-scale experiments on the Grid'5000
  testbed uptil 600 participants. As expected, we observe a
  logarithmic progression of the generated traffic according to the
  size of the shared document and the number of participants.
\end{abstract}

% If currently real-time editors is mainly used by small
%   groups, supporting large groups allow real-editing to be used in
%   large events such as massive online courses, conferences, TV-shows \ldots

  % Building a fully decentralized, scalable real-time editor requires
  % to find a difficult compromise between time, space, and
  % communication complexities. 


%%% Local Variables: 
%%% mode: latex
%%% TeX-master: "../paper"
%%% End: 
