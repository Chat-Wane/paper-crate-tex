
\begin{abstract}
  Distributed real-time editors such as Google Docs, or Etherpad allow users
  distributed in space and organizations to collaborate easily with only web
  browsers. Yet, main stream editors rely on central servers with issues in
  privacy, single point of failure and scalability.  Decentralized editors
  settle privacy problems but scalability issues remain.  None of existing
  algorithms provide satisfying complexities in both communication and space.
  In this paper, we provide the complexity analysis of \LSEQ, an algorithm the
  result of which enjoys a polylogarithmic upper bound on space complexity. In
  turns, it directly impacts traffic. We provide the outline to build a
  decentralized real-time editor. We evaluate our editor \CRATE on large scale
  experiments involving up till 600 participants. The results show a traffic
  increasing as $\mathcal{O}((\log I)^2\ln R)$ where $I$ is the number of
  insertions in the document, and $R$ the number of participants. As such, the
  editor scales well in both these dimensions.

  % In this paper, we propose \CRATE; a working decentralized real-time editor
  % that supports large number of users working on large documents. In particular,
  % \CRATE achieves a sublinearly upper-bounded communication complexity in
  % $\mathcal{O}((\log d)^2\ln{m})$ where $d$ is the document size and $m$ the
  % membership size. Thanks to WebRTC, \CRATE can be easily deployed on a network
  % of browsers.  We evaluate \CRATE on large-scale experiments in the Grid'5000
  % testbed involving uptill 600 participants. As expected, we observe a
  % logarithmic progression of the generated traffic according to the size of the
  % shared document and the number of participants.
\end{abstract}

%%% Local Variables:
%%% mode: latex
%%% TeX-master: "../paper"
%%% End:
