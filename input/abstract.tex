

\begin{abstract}
% I followed
% http://www.easterbrook.ca/steve/2010/01/how-to-write-a-scientific-abstract-in-six-easy-steps/ to structure the abstract

  Distributed real-time editors such as google docs, etherpad allow
  people to work distributed in space and accross
  organization. However, current solutions either lack of scalablity
  or suffer of issues in privacy, censorship, single-point-of-failure,
  ease of deployement, economic intelligence, restriction in terms of
  service. Building a fully decentralized, scalable, easy to deploy
  real-time editor requires to find a difficult compromise between
  time, space, and traffic complexity with respect to the number of
  users and size of shared documents. In this paper, we took advantage
  of WebRTC to build \CRATE; an easy-to-deploy real-time editor that
  runs on a network of browsers i.e. without servers. We propose a new
  class of complexity in order to support large number of users
  working on large document. We extensively describe each core
  components, we provide the proof of the polylarithmic upper-bound on
  the space complexity of the shared sequence. We evaluate \CRATE in
  large-scale experiments on the Grid'5000 testbed until 500
  participants. We observed logarithmic progession of generated
  traffic according the size of shared document.

\end{abstract}

\begin{abstract}
  The recent WebRTC technology has renewed the interest over large-scale
  distributed web applications. Over the last decades, the number of users of
  distributed collaborative editors like Google Docs, Etherpad, or CoWeb, has
  not ceased to increase. They allow distributing the work across space, time,
  and organizations to create better documents. 

  However, current solutions either lack of scalablity or suffer of
  issues in privacy, censorship, single-point-of-failure, economic
  intelligence, restriction in terms of service. 

  In this paper, we address these issues by introducing \CRATE, a
  distributed and decentralized collaborative editor running in web
  browsers.  As core components, \CRATE uses \LSEQ to efficiently
  represent its shared document, and \SPRAY to automatically adjust
  the editing session membership.  In this paper, we extensively
  describe each core components, we provide the proof of the
  polylarithmic upper-bound on the space complexity of the shared
  sequence. We evaluate \CRATE in large-scale experiments on the
  Grid'5000 testbed. Being scalable and decentralized, it constitutes
  a serious competitor to current trending editors.

  % The recent development distributed collaborative editors has renewed the
  % interest for efficient data structures that can allow a huge number of
  % simultaneous users. Indeed, existing editors lack scalability in terms of the
  % number of users, concurrency, and the size of the produced
  % documents. Additionally, centralised editors suffer from privacy issues,
  % single-point-of-failure, economic intelligence issues, and restrictions in
  % terms of service.  Several replicated data structures implementing sequences
  % have been proposed to alleviate most of the aforementioned issues. Yet, they
  % have an unsatisfying space complexity. Either they use tombstones and then
  % deleted elements are only hidden to the user or they allocate to each element
  % of the sequence an identifier that may be as large as the number of elements
  % in the sequence. This paper proposes an identifier allocation function named
  % \NAME{}. It provides identifiers enjoying a sub-linear space complexity
  % without being application specific and proves the polylogarithmic size of
  % these identifiers. Moreover, this paper provides all the outlines to develop
  % distributed collaborative editors that scale in all the aforementioned
  % dimensions. Using these outlines, we developed a prototype called
  % \EDITORNAME{} ({\sc c}ollabo{\sc rat}ive {\sc e}ditor) that we evaluated on
  % the Grid'5000 testbed. The results show that such editors, being
  % decentralised and scalable, could constitute a serious competitor to current
  % trending editors. In particular, it allows massive collaborative authoring of
  % huge documents which opens the field to new kinds of distributed
  % applications.
\end{abstract}

%%% Local Variables: 
%%% mode: latex
%%% TeX-master: "../paper"
%%% End: 
