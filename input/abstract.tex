\begin{abstract}
  % Distributed real-time editors such as Google Docs~\cite{nichols1995high}, or
  % Etherpad allow users distributed in space and organizations to collaborate
  % easily with only web browsers. Yet, main stream editors rely on central
  % servers with issues in privacy, single point of failure and scalability.
  % Decentralized editors settle privacy problems but scalability issues remain.
  % None of existing algorithms provide satisfying complexities in both
  % communication and space.  In this paper, we propose \CRATE; a working
  % decentralized real-time editor that supports large number of users working on
  % large documents. In particular, \CRATE achieves a sublinearly upper-bounded
  % communication complexity in $\mathcal{O}((\log d)^2\ln{m})$ where $d$ is the
  % document size and $m$ the membership size. Thanks to WebRTC, \CRATE can be
  % easily deployed on a network of browsers.  We evaluate \CRATE on large-scale
  % experiments in the Grid'5000 testbed involving uptill 600 participants. As
  % expected, we observe a logarithmic progression of the generated traffic
  % according to the size of the shared document and the number of participants.

  Distributed collaborative editors such as Google Docs, or Etherpad allow users
  to write documents together. To provide availability and responsiveness of
  documents, real-time editors use the optimistic replication. As such, each
  user creates a local copy of the document and directly performs her
  modification on it. Changes are broadcast to all replica owners where they are
  integrated. Strong eventual consistency states that replicas integrating an
  identical set of operations converge to an equivalent state, i.e., users read
  a same document.

  Yet, mainstream editors rely on central servers which bring issues in privacy,
  scalability, and single point of failure. Decentralized editors settle privacy
  problems but scalability issues remain. 

  Operational transformation (OT) approaches provide very efficient local
  execution of operations. Nevertheless, once the operations arrive to remote
  replicas, OT may transform each operation to retrieve consistent arguments
  taking into account the changes integrated since operation execution. The
  transformation cost heavily depends on concurrency. In addition, concurrency
  detection structures grow at least linearly compared to the number of
  writers. Hence, OT behave great in stable environments with small groups of
  users but should not be used in large scale system subject to unpredictable
  latency.

  To avoid concurrency costs, data types recently emerged providing commutative
  operations. Such data structures exist for counters, sets, graphs etc. In this
  paper, we focus on data structures for sequences -- the closest structure from
  documents.
  
  Unique and immutable identifiers enable commutativity of operations. Two
  families exist concerning the allocation of identifiers:


\end{abstract}

%%% Local Variables: 
%%% mode: latex
%%% TeX-master: "../paper"
%%% End: 
