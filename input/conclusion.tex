
\section{Conclusion}

\label{sec:conclusion}
In the context of collaborative editing, this paper proposed an identifier
allocation function \NAME{} for building sequence replicated data
structures. The obtained sequences enjoy good space and time complexities. The
proof on space complexity demonstrates logarithmic, polylogarithmic, and
quadratic for, respectively, random, monotonic, and worst-case
insertions. Using this allocation function and interval version vectors to
track causality, we developed a distributed collaborative editor called
\EDITORNAME{}. This editor is fully decentralised and scales in terms of the
number of peers, concurrency, and document size. We validated the approach on a
setup close of the real life conditions and using extreme parameters: low
startup values, high number of insertions, high number of peers and high
latency. The experiments confirmed the space complexity of \NAME{} and the
scalability of \EDITORNAME{}. As such, it alleviates the issues brought by
centralised approaches: single point of failure, privacy issues, economic
intelligence issues, limitations in terms of service, etc. Also, it alleviates
the scalability issues of decentralised approaches. As such, it can be seen as
a serious competitor for current trending editors (e.g. Google Docs,
SubEthaEdit,~\ldots), allowing massive collaborative editing without any
service providers. More specifically, it opens the field to a new range of
distributed applications such as massive editing of online courses, or
collaborative reviewing of co-located events, webinars etc.


%% Future work concerns the improvement of our prototype
%% CRATE: \begin{enumerate} \item We aim to develop of a truly tree-based
%% implementation of \NAME{}. Indeed, the current version uses a data structure
%% which is a linearisation of the tree. As a consequence, each element is
%% linked to its identifier without factorising the common parts of the
%% latter. Therefore, the memory usage is currently higher than the one
%% suggested by the theoretical analysis of Section~\ref{sec:proposal}. The
%% network load would remain unchanged.  \item Currently, CRATE only includes
%% the core functionalities described in this paper. However, we intend to add
%% common functionalities such as the modifications of document style, the
%% group awareness, etc.  \item Recent technological advances finally allow to
%% build peer-to-peer applications within the web browser without any
%% additional plug-ins. The real strength of current editors hosted by service
%% providers is their ease of access. Nevertheless, the aforementioned progress
%% puts distributed editors on an equal footing in that regard. Furthermore, it
%% extends the tool to a range of users equipped with small devices but still
%% embedding a web browser such as smartphone users, or Raspberry Pi users
%% etc.  \end{enumerate}

% Future work concerns the causality tracking issue. Indeed, the chosen
% trade-off proposed by the interval version vectors allows scaling in number
% of peers. Nevertheless, when the network is subject to churn, the local
% memory used to store the vector can grow quickly. We envision an approach
% trading accuracy for memory without sacrificing on correctness.

%%% Local Variables: 
%%% mode: latex
%%% TeX-master: "../paper"
%%% End: 
