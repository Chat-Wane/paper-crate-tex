
\section{Conclusion}
\label{sec:conclusion}

In this paper, we demonstrated that \LSEQ makes large-scale decentralized
editors a practicable alternative to mainstream editors. \LSEQ-based editors
such as \CRATE enable real-time editing at anytime, anywhere, whatever the number
of participants, without third parties.

We proposed an original tradeoff on time, space, and communication
complexities. The balance mainly consists in providing sublinear communications
while maintaining affordable memory consumption and efficient operations.  We
validated this new tradeoff on large-scale experiments involving up till 600 web
browsers interconnected.

Editors smoothly adapt their behavior to editing sessions without any global
knowledge. Thus, they become useful not only for small groups but also in large
events such as conferences, or massive online lectures, that can gather a large
number of participants.

For future work, we intend to study the behavior of users when they are part of
large editing sessions. Are current user interfaces descriptive enough to handle
such scenarios?  \CRATE allows clients to contribute to their editing sessions
by sharing their resources (e.g. bandwidth).  It makes real-time editing cheap
for any web applications, for web application providers do not spend further
resources. For instance, \CRATE could be embedded in Wikipedia to solve peaks of
concurrent changes resulting in conflicts.

Finally, we demonstrated how a well-known collaborative application can be
provided without collaboration providers. We aim to explore if other
collaborative applications such as distributed calendars or crowdsourcing
platforms can be deployed on a network of browsers.

%%% Local Variables: 
%%% mode: latex
%%% TeX-master: "../paper"
%%% End: 
