
\section{Conclusion}
\label{sec:conclusion}

In this paper, we demonstrated that \LSEQ makes large-scale decentralized
editors a practicable alternative to mainstream editors. \LSEQ-based editors
such as \CRATE enable real-time editing at anytime, anywhere, whatever the number
of participants, without third parties.

We proposed an original tradeoff on time, space, and communication
complexities. The balance mainly consists in providing sublinear communications
while maintaining affordable memory consumption and efficient operations.  We
validated this new tradeoff on large-scale experiments involving up till 600 web
browsers interconnected.

Editors smoothly adapt their behavior to editing sessions without any global
knowledge. Thus, they become useful not only for small groups but also in large
events such as conferences, or massive online lectures, that can gather a large
number of participants.

For future work, we intend to study the behavior of users when they are part of
large editing sessions. Are current user interfaces descriptive enough to handle
such scenarios?  \CRATE allows clients to contribute to their editing sessions
by sharing their resources (e.g. bandwidth).  It makes real-time editing cheap
for any web applications, for web application providers do not spend further
resources. For instance, \CRATE could be embedded in Wikipedia to solve peaks of
concurrent changes resulting in
conflicts. %% In this case, any Wikipedia article can be observed in real time.

Finally, we demonstrated how a well-known collaborative application can be
provided without collaboration providers. We aim to explore if other
collaborative applications such as distributed calendars or crowdsourcing
platforms can be deployed on a network of browsers.


%% In this paper, we described how it is possible to build a decentralized
%% collaborative editor that allow real-time editing anytime, anywhere, whatever
%% the number of participants.

% First, we demonstrated that a decentralized editor can be deployed by just
% clicking on a http link as for google doc. We made decentralized editors easy
% to use.

% Second, thanks to a new trade-off on different complexities, real-time sessions
% are established at affordable cost without a cloud provider. Hence, shared
% documents belongs only to participants without a collaboration provider.

% Third, we removed any limitations on the number of participants allowing
% real-time editing to be used not only for small groups but also in large
% events such as conferences, massive online lectures. CRATE can smoothly handle
% the transitions from small groups to large groups enlarging the affordance of
% real-time editing.

% Compared to previous work, we established a new tradeoff on
% communication, time and space complexities by first upper-bounding
% communication complexity to a sublinear complexity and keep time and space
% complexities in acceptable classes of complexities.  We validated
% experimentally this new tradeoff on a real working system publicly
% available.

% In perspective, we aim to study how users behave when editing in large
% group. Does current user interfaces can handle such scenario? 
% CRATE allow to provide real-time editing on client ressources. It
% makes real-time editing cheap for any web applications because it does
% not require additional ressources from web application provider. We
% aim to see how CRATE can be embedded in well-known applications such as
% Wikipedia. In this case, any wikipedia article can be observed in real
% time.

% Finally, we demonstrated how one well-known collaborative application
% can be provided without a collaboration provider. We aim to explore if
% other collaborative applications such as shared calendar or
% crowdsourcing platforms can deployed on a network of browsers.



% \section{Conclusion}
% \label{sec:conclusion}

% This paper provides the complexity analysis of the allocation function
% \LSEQ. Among others, \LSEQ enjoys logarithmic, polylogarithmic, and quadratic
% growth of identifiers for respectively, random, monotonic, and worst-case
% insertions. The worst-case is made non-trivial using two sub-allocation
% functions with antagonist designs. The experiments validates the complexity
% analysis in space and time.  Using \LSEQ, decentralized collaborative editors
% scale in terms of number of insertions performed on the document. This paper
% provided the outlines to build such editor. In particular, we introduced \CRATE
% as the composition of \LSEQ, \SPRAY, and a version vector with exceptions. Such
% composition allows large scale editing of documents in real-time. The
% experiments involving up till 600 browsers confirmed the scalability of such
% composition. Since it settles privacy issues brought by centralized solutions,
% and alleviates scalability issues, such editor competes with current trending
% editors such as Google Docs, Etherpad, or SubEthaEdit. In particular, it allows
% massive collaborative editing of large documents without service
% providers. Hence, it opens the field to a new range of distributed applications
% such as massive editing of online courses, collaborative reviewing of co-located
% events, or webinars.



% In the context of collaborative editing, this paper proposed an identifier
% allocation function \LSEQ for building sequence replicated data structures. The
% obtained sequences enjoy good space and time complexities. The proof on space
% complexity demonstrates logarithmic, polylogarithmic, and quadratic for
% respectively, random, monotonic, and worst-case insertions.  Using this
% allocation function and interval version vectors to track causality, we
% developed a distributed collaborative editor called \CRATE. This editor is fully
% decentralised and scales in terms of the number of peers, concurrency, and
% document size. We validated the approach on a setup close of the real life
% conditions and using extreme parameters: low startup values, high number of
% insertions, high number of peers and high latency. The experiments confirmed the
% space complexity of \LSEQ and the scalability of \CRATE. As such, it alleviates
% the issues brought by centralised approaches: single point of failure, privacy
% issues, economic intelligence issues, limitations in terms of service,
% etc. Also, it alleviates the scalability issues of decentralised approaches. As
% such, it can be seen as a serious competitor for current trending editors
% (e.g. Google Docs, SubEthaEdit,~\ldots), allowing massive collaborative editing
% without any service providers. More specifically, it opens the field to a new
% range of distributed applications such as massive editing of online courses, or
% collaborative reviewing of co-located events, webinars etc.


%% Future work concerns the improvement of our prototype
%% CRATE: \begin{enumerate} \item We aim to develop of a truly tree-based
%% implementation of \LSEQ. Indeed, the current version uses a data structure
%% which is a linearisation of the tree. As a consequence, each element is
%% linked to its identifier without factorising the common parts of the
%% latter. Therefore, the memory usage is currently higher than the one
%% suggested by the theoretical analysis of Section~\ref{sec:proposal}. The
%% network load would remain unchanged.  \item Currently, CRATE only includes
%% the core functionalities described in this paper. However, we intend to add
%% common functionalities such as the modifications of document style, the
%% group awareness, etc.  \item Recent technological advances finally allow to
%% build peer-to-peer applications within the web browser without any
%% additional plug-ins. The real strength of current editors hosted by service
%% providers is their ease of access. Nevertheless, the aforementioned progress
%% puts distributed editors on an equal footing in that regard. Furthermore, it
%% extends the tool to a range of users equipped with small devices but still
%% embedding a web browser such as smartphone users, or Raspberry Pi users
%% etc.  \end{enumerate}

% Future work concerns the causality tracking issue. Indeed, the chosen
% trade-off proposed by the interval version vectors allows scaling in number
% of peers. Nevertheless, when the network is subject to churn, the local
% memory used to store the vector can grow quickly. We envision an approach
% trading accuracy for memory without sacrificing on correctness.

%%% Local Variables: 
%%% mode: latex
%%% TeX-master: "../paper"
%%% End: 
