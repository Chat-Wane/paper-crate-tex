
\section{Introduction}

Google Docs or Etherpad allow millions of users to start and access real-time
editing sessions very easily by the mean of URLs in web browsers. Yet, such
editors require collaboration providers to setup servers and enable
collaboration. It implies issues in privacy, censorship, and economic
intelligence. It also implies scalability issues in term of number of
participants. Despite that current editors limit \emph{de facto} the groups
size, emerging contexts (e.g. massive online courses, conferences, TV-shows)
expose the needs of larger collaboration groups.  For instance, writing notes on
Google Docs is possible uptill fifty users. Next users have their rights limited
to the reading of the document.

Decentralized real-time editors~\cite{oster2006data, sun1998operational,
  sun2009contextbased} do not require intermediate servers and by the same solve
privacy-related issues. However, scalability issues remain.  Addressing
scalability issues requires finding a good trade-off between communication,
space and time complexities. Such trade-off mainly consists in a sublinear
communication complexity compared to the editing session size. While the
transmission of operations requires at least a multiplicative factor
logarithmically scaling compared to the network size, maintaining consistency
requires piggybacking additional metadata the complexity of which must be
carefully studied.

Decentralized algorithms that belong to operational
transformation~\cite{sun2009contextbased} approach require piggybacking a state
vector in order to detect concurrent operations. Unfortunately, state vectors
grow linearly compared to the number of members that ever participated in the
authoring.

Conflict-Free Replicated Data Types~\cite{shapiro2011comprehensive} (CRDTs) such
as WOOT~\cite{oster2006data} piggyback a constant-size identifier which is
optimal. Nonetheless, their local space complexity grows monotonically which
eventually leads to the need of an unaffordable garbage collecting. Other
algorithms~\cite{preguica2009commutative, weiss2010logootundo} limit their local
space consumption. Nevertheless, they piggyback variable-size identifiers that
can grow linearly compared to the document size. In such case, they eventually
need to run a costly consensus algorithm to rebalance
documents~\cite{zawirski2011asynchronous}. Finally, the \LSEQ
algorithm~\cite{nedelec2013concurrency} aims to avoid such consensus by
sublinearly upper-bounding the space complexity of its identifiers. \TODO{It
  conjectured that identifiers can be bounded to the $log(s)^2$ where $s$ is the
  size of shared document.}

In this paper, we propose \CRATE, a scalable real-time editor that runs on a
network of browsers. \CRATE relies on WebRTC and browser-to-browser
communication to provide an easy access for end-users. Compared to
state-of-the-art, it presents a different tradeoff which balances the load
between space, time, and communication complexities. The contributions of this
paper are threefold:
\begin{itemize}
\item Compared to previous work~\cite{nedelec2013lseq}, we demonstrate the upper
  bounds on space and time complexities of \LSEQ. In addition to which we
  provide the communication complexity of \CRATE. We shows the balance in these
  dimensions and we position it in relation to state-of-the-art.
\item We describe \CRATE's internal components. In particular, we detail \LSEQ
  that manages the shared document, and \SPRAY that creates the network of
  browsers. Our Javascript implementation of \CRATE runs directly in browsers
  without any plugins.
\item We conducted experimental studies to validate the complexities of
  \CRATE. The experiments took place in the Grid'5000 testbed and involved
  uptill $600$ real web browsers opened to edit a shared document. At a rate of
  100 insertions per seconds, the document size reached above 1 million
  characters. As expected, we observed a logarithmic growth of the traffic
  compared to the number of participants, and a polylogarithmic growth of the
  traffic compared to the size of the document.
\end{itemize}

The remainder of this paper is organized as follows:
Section~\ref{sec:relatedwork} reviews the related work with an emphasis on the
complexity tradeoff proposed by other approaches. Section~\ref{sec:proposal}
follows with a description of \CRATE and its core
components. Section~\ref{sec:experiments} highlights the scalability of \CRATE
while validating the complexity analysis of \LSEQ and \SPRAY. Finally,
Section~\ref{sec:conclusion} concludes the paper and discusses about
perspectives.

%%% Local Variables: 
%%% mode: latex
%%% TeX-master: "../paper"
%%% End: 
