
\section{Introduction}

Real-time collaborative editors are used by millions of users and allow them to
distribute the work over locations and across organizations.

However, the most well-known editors such as Google Docs rely on central
servers. The topology itself implies issues in privacy, censorship and economic
intelligence. Indeed, since all communications transit a server, the owner of
the latter is able to intercept and use the information without the users'
consent (REF).  It also implies restrictions in terms of service. For instance,
Google Docs restricts the writing access of its documents to 50 users
simultaneously. Beyond, the users are readers. Such limitations preclude the
existence of crowd edited documents which could be useful in educational cases.
\TODO{moar}.

Recently, WebRTC allowed decentralized applications to run directly inside web
browsers. Hence, they become as easy-to-deploy as centralized solutions and do
not require any collaboration providers. Many algorithms~\cite{oster2006data,
  sun1998operational, sun2009contextbased} exist proposing different tradeoff
between complexities in space, time and communications. Yet, finding a suitable
compromise for browsers remains challenging since the range of users includes
small devices such as smartphones with limited memory, CPU, and bandwidth.

The most difficult part that distributed algorithms have to tackle concerns
concurrent operations, i.e., when multiple users perform changes on a document
simultaneously. Users can experience different results after receipt of
operations. To solve this, former algorithms~\cite{} are transforming the
operations against concurrent ones in order to properly adapt their arguments.
Unfortunately, these approaches do not scale in number of users (communication
and memory) and concurrency (time at integration). Latter algorithms~\cite{}
propose sequence data types providing commutative operations. Therefore, the
integration order of concurrent operations does not matter since they will
converge to an identical structure. However, the first class of these shared
sequences uses tombstones which simply hides removed elements. Consequently,
they do not scale number of operations (space and time). The second class of
these shared sequences uses identifiers that have variable sizes. Unfortunately,
the editing behavior (i.e., the position of insertions in the document) can
badly impact the size of these identifiers (space, time, and communication).

In this paper, we propose \CRATE, a real-time editor that runs on a network of
browsers. \CRATE relies on WebRTC and browser-to-browser communication to
provide an easy access for end-users. Compared to state-of-art, \CRATE proposes
a new complexity tradeoff between time, space, and communication that fits web
constraints in order to run within browsers.

This main contributions of this paper are three-fold:
\begin{itemize}
\item we describe \CRATE's system and its core-components, namely \LSEQ and
  \SPRAY. \CRATE's Javascript implementation runs directly inside browsers
  without any plugins.
\item Compared to previous work~\cite{nedelec2013lseq}, we demonstrate the upper
  bound on space complexity of \LSEQ and we analyze the complete complexity of
  \CRATE compared to state of art.
\item We conducted experimental studies to validate the complexities of the
  \CRATE's implementation inside real web browsers. From $2$ to $600$
  participants editing a document reaching more than $1$ million characters, we
  observed a logarithmic growth of the traffic compared to the number of
  participants, and a polylogarithmic growth of the traffic compared to the size
  of the document.
\end{itemize}

The remainder of this paper is organized as
follows. Section~\ref{sec:relatedwork} reviews the related work with an emphasis
on the complexity tradeoff proposed by other
approaches. Section~\ref{sec:proposal} follows with a description of \CRATE and
its core components. Section~\ref{sec:experiments} highlights the scalability of
\CRATE while validating the space complexity analysis of \LSEQ and
\SPRAY. Finally, Section~\ref{sec:conclusion} concludes the paper and discusses
about perspectives.

%%% Local Variables: 
%%% mode: latex
%%% TeX-master: "../paper"
%%% End: 
