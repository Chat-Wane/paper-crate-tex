
\section{Introduction}

Google Docs or Etherpad allow millions of users to start and access real-time
editing sessions very easily by the mean of URLs in web browsers. Yet, such
editors require collaboration providers to setup servers and enable
collaboration. It implies issues in privacy, censorship, and economic
intelligence. It also implies scalability issues in term of number of
participants. Despite that current editors limit \emph{de facto} the groups
size, emerging contexts (e.g. massive online courses, conferences, TV-shows)
expose the needs of larger collaboration groups.  For instance, writing notes on
Google Docs is possible uptill fifty users. Next users have their rights limited
to the reading of the document.

Decentralized real-time editors~\cite{oster2006data, sun1998operational,
  sun2009contextbased} do not require intermediate servers and by the same solve
privacy-related issues. However, scalability issues remain.  Addressing
scalability issues requires finding a good trade-off between communication,
space and time complexities. Such trade-off mainly consists in a sublinear
communication complexity compared to the editing session size. While the
transmission of operations requires at least a multiplicative factor
logarithmically scaling compared to the network size, maintaining consistency
requires piggybacking additional metadata the complexity of which must be
carefully studied.

Decentralized algorithms that belong to operational
transformation~\cite{sun2009contextbased} require piggybacking a state vector in
order to detect concurrent operations. Unfortunately, state vectors grow
linearly compared to the number of members that ever participated in the
authoring.

Conflict-Free Replicated Data Types~\cite{shapiro2011comprehensive} (CRDTs) such
as WOOT~\cite{oster2006data} piggyback a constant-size identifier which is
optimal. Nonetheless, their local space complexity grows monotonically which
eventually leads to the need of an unaffordable garbage collecting. Other
algorithms~\cite{weiss2010logootundo, preguica2009commutative} limit their local
space consumption. Nevertheless, they piggyback variable-size identifiers that
can grow linearly compared to the document size. In such case, they eventually
need to run a costly consensus algorithm to rebalance
documents~\cite{zawirski2011asynchronous}. Finally, the \LSEQ
algorithm~\cite{nedelec2013concurrency} aims to avoid such consensus by
sublinearly upper-bounding the space complexity of its identifiers. \TODO{It
conjectured that identifiers can be bounded to the $log(s)^2$ where $s$ is the
size of shared document.}

In this paper, we propose \CRATE, a scalable real-time editor that runs on a
network of browsers. \CRATE relies on WebRTC and browser-to-browser
communication to provide an easy access for end-users. \CRATE belongs to
Conflict-Free Replicated Datatypes (CRDT) class of real-time editors. Compared
to state of art, we reduced space complexity of a shared document from linear to
polylogarithmic whithout changing time and space complexities of this class of
editors.  The contributions of this paper are three-fold:
\begin{itemize}
\item Compared to previous work~\cite{nedelec2013lseq}, we demonstrate
  the upper bound on space complexity of \LSEQ in case of monotonic
  editings and we analyze the complete complexity of \CRATE compared
  to state of art.
\item We describe \LSEQ for managing the shared document
  and \SPRAY a gossip algorithm based on WEBRTC. \CRATE's Javascript
  implementation runs directly inside browsers without any plugins.
\item We conducted experimental studies to validate the complexities of the
  \CRATE's implementation inside real web browsers. As expected, from $2$ to $600$
  participants editing a document reaching more than $1$ million characters, we
  observed a logarithmic growth of the traffic compared to the number of
  participants, and a polylogarithmic growth of the traffic compared to the size
  of the document.
\end{itemize}

The remainder of this paper is organized as
follows. Section~\ref{sec:relatedwork} reviews the related work with an emphasis
on the complexity tradeoff proposed by other
approaches. Section~\ref{sec:proposal} follows with a description of \CRATE and
its core components. Section~\ref{sec:experiments} highlights the scalability of
\CRATE while validating the space complexity analysis of \LSEQ and
\SPRAY. Finally, Section~\ref{sec:conclusion} concludes the paper and discusses
about perspectives.

%%% Local Variables: 
%%% mode: latex
%%% TeX-master: "../paper"
%%% End: 
