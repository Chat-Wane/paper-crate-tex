
\section{Introduction}

Distributed real-time editors such as google docs, etherpad allow
people to work distributed in space and accross organization and are
now used by milllions of users. 

However, main stream real-time editors such google doc relies on
centralized servers with issues in privacy, censorship or economic
intelligence. It has also restrictions in terms of service i.e. google
doc is restricted to 50 real-time users. 
%
Real-time editors can be decentralized requiring no collaboration
providers. Many algorithms has been proposed~\cite{sun1998operational,
  sun2009contextbased,woot} but it is hard to find a good comprise for
a full working system with complexities in time, space and
communications. due to the communication layer setup, It also
difficult to achieve for a decentralized editor a 'one-click'
deployement as Google doc can do.

In this paper, we propose \CRATE, a real-time editor that runs on a
network of browsers. \CRATE relies on WEBRTC and browser-to-browser
connections to ensure easy deployement for end-users. Compared to
state of art, \CRATE propose a new class of complexity for time,
space, communication that better fit web constraints in order to run
within browsers.

This main constribution of this paper are three-fold:
\begin{itemize}
\item An in-depth description of \CRATE and its core-components,
  namely \LSEQ and \SPRAY, with a full implementation in pure
  javascript than run in browsers.
\item Compared to previous work~\cite{doceng}, we
  demonstrate that the upper bound space complexity of \CRATE.
\item Experimental studies that validate the space complexity of \LSEQ and the
  scalability of \CRATE. Parts of theses experiments ran on the Grid'5000
  testbed and involved up to $500$ emulated peers creating a document of more
  than one million characters at a global rate of one hundred operations per
  second.
\end{itemize}

The remainder of this paper is organized as
follows. Section~\ref{sec:preliminaries} provides the necessary background to
sequence data structures and the motivations. Section~\ref{sec:proposal}
follows with a description of \LSEQ and the proof of its space
complexity. Section \ref {sec:experiments} highlights the scalability of
\CRATE while validating the space complexity analysis of
\LSEQ. Section~\ref{sec:relatedwork} reviews related works. Finally,
Section~\ref{sec:conclusion} concludes the paper.




% The recent WebRTC technology has opened a new playground for large-scale
% distributed web applications. Among others, distributed collaborative editors
% such as Google Docs, SubEthaEdit, or Etherpad, have proven their
% usefulness. Such editors allow distributing the work across space, time and
% organizations. However, they have several limitations that preclude the
% existence of a massive distributed collaborative editor. 

% % due either to
% % centralisation (e.g. single-point-of-failure, privacy issues, economic
% % intelligence issues, limitations in terms of service) or to decentralisation
% % (e.g. number of users, concurrency and synchronisation, number of
% % operations). For instance Google~\cite{nichols1995high} allows only 50 users to
% % write a document together at a same time. These scalability issues preclude the
% % existence of a massive distributed collaborative editor.

% Distributed collaborative editors use the optimistic replication
% scheme~\cite{johnson1975maintenance, saito2005optimistic} to provide high
% availability and responsiveness. Locally, each user involved in the
% collaboration owns a copy of the document (replica). Local operations are
% executed and become effective locally before being broadcast. Remotely, each
% user integrates the received operations to her replica. The convergence property
% of optimistic replication states that replicas are allowed to temporarily
% diverge. Yet, the resulting replicas become identical when all collaborators
% have received all changes~\cite{demers1987epidemic}.  Thus, optimistic
% replication requires
% \begin{inparaenum}[(i)]
% \item a shared data type that ensures
%   consistency and
% \item an algorithm to disseminate changes.
% \end{inparaenum}

% As shared data type, most of state-of-the-art editors use operational
% transformations~\cite{sun1998operational, sun2009contextbased}
% (OT). Unfortunately, the time complexity of the integration of operations is
% quadratically \TODO{upper}-bounded by the number of concurrent operations
% relatively to the integrated one. In other terms, all collaborators suffer from
% few concurrent operations making OT approaches non practical because of
% \TODO{unpredictable latency}.

% For the information dissemination, \TODO{most} of distributed collaborative
% editors (REF) running as web applications rely on a central server to spread the
% operations. The topology itself brings single-point-of-failure, privacy issues,
% economic intelligence issues, limitations in terms of service, and lack of
% scalability.

% In this paper, we introduce \CRATE, a distributed and decentralized
% collaborative editor running in web browsers. It uses a family of distributed
% data structure called Conflict-free Replicated Data
% Type~\cite{shapiro2011comprehensive, shapiro2011conflict} (CRDT) well-suited to
% handle concurrency. To ensure convergence, a unique and immutable identifier is
% associated with each element (e.g. character, words, line, paragraph) in the
% sequence (i.e. document). \CRATE uses \LSEQ~\cite{nedelec2013concurrency,
%   nedelec2013lseq} to allocate identifiers enjoying a sub-linear upper-bound on
% space complexity compared to the number of insertions in the document. To
% propagate the operations, \CRATE uses \SPRAY, a random peer sampling protocol
% that automatically adjusts the neighborhood of each peer to reflect the editing
% session membership. As such, it efficiently disseminates the changes performed
% to the document, it is resilient to non-correlated failures, it balances the
% load among peers etc.

% Contribution of this paper are:
% \begin{itemize}
% \item An in-depth description of \CRATE and its core-components, namely \LSEQ
%   and \SPRAY.
% \item Experimental studies that validate the space complexity of \LSEQ and the
%   scalability of \CRATE. Parts of theses experiments ran on the Grid'5000
%   testbed and involved up to $500$ emulated peers creating a document of more
%   than one million characters at a global rate of one hundred operations per
%   second.
% \end{itemize}

% The remainder of this paper is organized as
% follows. Section~\ref{sec:preliminaries} provides the necessary background to
% sequence data structures and the motivations. Section~\ref{sec:proposal}
% follows with a description of \LSEQ and the proof of its space
% complexity. Section \ref {sec:experiments} highlights the scalability of
% \CRATE while validating the space complexity analysis of
% \LSEQ. Section~\ref{sec:relatedwork} reviews related works. Finally,
% Section~\ref{sec:conclusion} concludes the paper.

%%% Local Variables: 
%%% mode: latex
%%% TeX-master: "../paper"
%%% End: 
