
\section{Introduction}

In recent years, the interest in distributed collaborative editors such as
Google Docs, SubEthaEdit, or Etherpad, has not stopped growing. Such editors
allow distributing the work across space, time and organisations. However,
despite being undoubtedly useful, they have several limitations due either to
centralisation (e.g. single-point-of-failure, privacy issues, economic
intelligence issues, limitations in terms of service) or to decentralisation
(e.g. number of users, concurrency and synchronisation, number of
operations). For instance Google~\cite{nichols1995high} allows only 50 users to
write a document together at a same time. These scalability issues preclude the
existence of a massive distributed collaborative editor.

Decentralised distributed collaborative editors use the optimistic replication
scheme \cite{johnson1975maintenance, saito2005optimistic} to provide high
availability and responsiveness. Locally, each user involved in the
collaboration owns a copy of the document (replica). Local operations are
executed and become effective locally before being broadcast. Remotely, each
user integrates the received operations to her replica. The convergence
property of optimistic replication states that replicas are allowed to
temporarily diverge. Yet, the resulting replicas become identical when all
collaborators have received all changes~\cite{demers1987epidemic}.

Operational transformation (OT) approaches are the first techniques used to
build distributed collaborative editors. However, the time complexity of the
integration of operations is upper-bounded by $O(H^2)$ where $H$ is the number
of concurrent operations in the logfile relatively to the integrated
one. Consequently, all collaborators suffer from few concurrent operations
making OT approaches non practical because of unpredictable latency.

On the other side, a family of distributed data structure called Conflict-free
Replicated Data Type \cite{shapiro2011comprehensive,shapiro2011conflict} (CRDT)
has been introduced. This family encompasses data structures such as counters
and variants of sequences, sets, and graphs. This family does not include data
structures such as a register, a stack, etc. Fortunately, a sequence is a data
structure well-suited for building distributed collaborative editors that scale
in presence of concurrent operations. Contrarily to OT, most of the processing
time is located in the local part of operations, making this approach more
practical since each operation leads to $N$ remote integrations ($N$ being the
number of collaborators). To ensure convergence, a unique and immutable
identifier is associated with each element (e.g. character, words, line,
paragraph) in the sequence (i.e. document). The set of identifiers is paired
with a total order, and this latter actually makes the sequence.

Nevertheless, these sequence data structures have a space complexity
issue. Indeed, they provide two update operations (insert and delete) and use
an identifier allocation function that either depends on the type of the
performed operation
\cite{ahmed2011evaluating,conway2014language,grishchenko2010deep,oster2006data,preguica2009commutative,roh2011replicated,weiss2007wooki,wu2010partial,Yu2012stringwise},
or the arguments of the operation~\cite
{preguica2009commutative,andre2013supporting,weiss2009logoot}. In both cases,
the space overhead induced by the identifiers can be prohibitively high and
directly impacts performance. Unfortunately, this allocation is a crucial but a
non-trivial problem.

In this paper, we study sequence data structures \footnote{A short version of
  this paper appeared in \cite{nedelec2013lseq,nedelec2013concurrency}.} at the
core of which there is the identifier allocation function that provides the
different collaborators with identifiers they assign to the elements of the
associated sequence. This paper has four main contributions:
\begin{itemize}
\item A new identifier allocation function called \NAME{}.
\item A proof of the polylogarithmic space complexity of \NAME{}.
\item All the outlines to develop a distributed collaborative editor for
  massive editing of large documents and a prototype called \EDITORNAME{} (for
  {\sc c}ollabo{\sc rat}ive {\sc e}ditor).
\item A experimental study that validates the space complexity of \NAME{} and
  the scalability of \EDITORNAME{}. Parts of theses experiments ran on the
  Grid'5000 testbed and involved up to $450$ emulated peers creating a document
  of half a million characters at a global rate of $166$ operations per second.
\end{itemize}

The remainder of this paper is organised as
follows. Section~\ref{sec:preliminaries} provides the necessary background to
sequence data structures and the motivations. Section~\ref{sec:proposal}
follows with a description of \NAME{} and the proof of its space
complexity. Section \ref {sec:experiments} highlights the scalability of
\EDITORNAME{} while validating the space complexity analysis of
\NAME{}. Section~\ref{sec:relatedwork} reviews related works. Finally,
Section~\ref{sec:conclusion} concludes the paper.

%%% Local Variables: 
%%% mode: latex
%%% TeX-master: "../paper"
%%% End: 
