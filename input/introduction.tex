

\section{Introduction}

Google Docs made real-time editing in browsers easy for millions of
users. However, Google mediates real-time editing sessions with central servers
raising issues on privacy, censorship, and economic intelligence. It also raises
scalability issues in terms of the number of participants.  Despite that small
groups currently constitute the main range of users, events such as massive
online lectures, TV shows, conferences gather larger
groups~\cite{breslow2013studying}.  Google Docs supports large groups but only
the first fifty users can edit, next users have their rights limited to document
reading, not even in real-time.  We think that real-time editors should allow
editing at anytime and anywhere, whatever is the number of participants. Even if
only a small subset among millions of participants are writing, all participants
of the editing session should be able to read and write in real-time whenever
they want.

Decentralized real-time editors~\cite{oster2006data, sun1998operational,
  sun2009contextbased} do not require intermediate servers and by the same
settle privacy issues. However, scalability issues remain.  Addressing
scalability requires finding a good trade-off between communication, space and
time complexities. Among others, achieving a sublinear communication complexity
compared to the number of participants is crucial for supporting large groups.

To provide availability of documents and responsiveness, real-time editors use
optimistic replication~\cite{saito2005optimistic} of sequences, text documents
being sequences of characters. As such, each user hosts a local copy of the
document and directly performs her modifications on it. Changes are broadcast to
all replica owners where they are integrated. The system is correct iff the
replicas integrating an identical set of operations converge to an equivalent
state, i.e. users read a same document~\cite{shapiro2011conflict}.

Decentralized algorithms of Operational
Transformation~\cite{sun2009contextbased} (OT) piggyback a state or a context
vector in order to detect concurrent operations. Unfortunately, state vectors
grow linearly with respect to the number of members that ever participated in
the authoring. Hence, these approaches are efficient for small groups of users,
but poorly scale to highly dynamic large groups of users and high network
latency.

Conflict-Free Replicated Data Types~\cite{shapiro2011comprehensive} (CRDTs) do
not pay the price of concurrency detection by providing commutative
operations. However, they piggyback unique and immutable identifiers on every
operation broadcast on the network. The size of this identifier is the key for
the communication complexity and consequently the scalability of the
approach. Two classes of CRDTs designed for sequences exist:
\begin{itemize}[noitemsep, leftmargin=*]
\item  CRDTs such as WOOT~\cite{oster2006data} piggyback a constant-size
  identifier which is optimal. However, it requires to keep tombstones, i.e., it
  marks elements as removed and hide them from the users. An empty document may
  store thousands of deleted elements. Tombstones degrade integration time of
  operations. This integration time blocks the user and eventually makes the
  real-time editor unusable. Removing tombstones requires garbage collecting
  algorithms that do not scale~\cite{abdullahi1998garbage}.
\item CRDTs such as Logoot~\cite{weiss2010logootundo} do not require tombstones
  but piggyback variable-size identifiers. Depending on the identifier
  allocation strategy, the size of identifiers may grow linearly with the
  document size, increasing the communication cost. Balancing the structure
  requires consensus algorithms that do not
  scale~\cite{mostefaoui2015signature}.

  In previous works, we proposed an identifier allocation strategy called
  \LSEQ~\cite{nedelec2013concurrency, nedelec2013lseq}. It aims to avoid such
  balancing mechanism by sublinearly upper-bounding the space complexity of its
  identifiers. It is conjectured to have a polylogarithmic growth of the size of
  the identifiers $\mathcal{O}((\log I)^2)$ where $I$ is the number of
  insertions performed on the document.
\end{itemize}

\noindent The contributions of this paper are threefold:
\begin{itemize}[noitemsep, leftmargin=*]
\item To improve our previous works~\cite{nedelec2013concurrency,
    nedelec2013lseq}, we prove the upper bounds on space and time complexities
  of \LSEQ and the conditions upon which they apply. This result opens the way
  for building decentralized and scalable real-time editors.
\item To measure the overall performance of a real working system, we built
  \CRATE: a real-time decentralized editor running in web
  browsers~\cite{nedelec2016crate}. Compared to Google Docs, \CRATE enables
  real-time editing whatever the number of participants. To better preserve
  privacy of users, \CRATE ensures that shared documents only exist within
  browsers participating to the editing session. In this paper, we describe the
  complete architecture of \CRATE and review the contribution of each component
  to the scalability of the system. %% analyse the complexity of each component.
\item To validate our complexity analysis, we provide experiments characterizing
  \LSEQ's behavior and we compare it to state-of-the-art approaches. Results
  show that \LSEQ achieves an original tradeoff between time and space
  complexities. In the Grid'5000 testbed, we launched \CRATE editing sessions
  involving up till 600 connected web browsers. The generated traffic inherits
  from the scalability of both \LSEQ and messages dissemination. The resulting
  documents reach millions of characters. It validates the scalability of a full
  working editor built on top of \LSEQ.
\end{itemize}

The remainder of this paper is organized as follows:
Section~\ref{sec:relatedwork} reviews the related work.
Section~\ref{sec:preliminaries} describes the necessary background to understand
the replicated data types for sequences. Section~\ref{sec:proposal} describes
the principle and functioning of \LSEQ. In particular, it provides its
complexity. Section~\ref{sec:editor} describes the architecture of decentralized
real-time editors. The experiments of Section~\ref{sec:experiments} highlight
its scalability while validating the complexity analysis.
Section~\ref{sec:conclusion} concludes the paper.

%%% Local Variables: 
%%% mode: latex
%%% TeX-master: "../paper"
%%% End: 
