
\section{Introduction}

%Real-time collaborative editors are used by millions of users and allow them to
%distribute the work over locations and across organizations.


Google Doc or Etherpad allow millions of users to start a real-time
editing session by just clicking on a URL in their web
browsers. However, main real-time editors require collboration
providers to setup servers to enable
collaboration. This implies issues in privacy, censorship and economic
intelligence, single point of failure and
scalability.% . Central servers also introduce single point of
% failures and scalability issues.  For instance, Google Docs restricts
% the writing access of its documents to 50 users
% simultaneously. Beyond, the users are readers. Such limitations
% preclude the existence of crowd edited documents which could be useful
% in educational cases.  \TODO{moar}.
Decentralized real-time editors ~\cite{oster2006data,
  sun1998operational, sun2009contextbased} do not require intermediate
servers but scalability issues remains. According to the different
approaches, linear or polynomial complexities exists on time, space or
communications. This prevents decentralized editors to scale to many
users or on large documents. If currently real-time editors is mainly
used by small groups, supporting large groups occurs in new contexts
such as massive online courses, conferences, TV-shows where
people want to share their notes. Google doc supports large groups but
currently only fifty first users can edit in real-time, next users can
only view the shared document without editing. We think that
decentralized editors should be able to allow anybody to edit anytime
whatever the number of participant.

% Pascal: keep that for related work...
% A scalable real-time 
% The most difficult part that distributed algorithms have to tackle concerns
% concurrent operations, i.e., when multiple users perform changes on a document
% simultaneously. Users can experience different results after receipt of
% operations. To solve this, former algorithms~\cite{} are transforming the
% operations against concurrent ones in order to properly adapt their arguments.
% Unfortunately, these approaches do not scale in number of users (communication
% and memory) and concurrency (time at integration). Latter algorithms~\cite{}
% propose sequence data types providing commutative operations. Therefore, the
% integration order of concurrent operations does not matter since they will
% converge to an identical structure. However, the first class of these shared
% sequences uses tombstones which simply hides removed elements. Consequently,
% they do not scale number of operations (space and time). The second class of
% these shared sequences uses identifiers that have variable sizes. Unfortunately,
% the editing behavior (i.e., the position of insertions in the document) can
% badly impact the size of these identifiers (space, time, and communication).

In this paper, we propose \CRATE, a scalable real-time editor that runs on a network of
browsers. \CRATE relies on WebRTC and browser-to-browser communication to
provide an easy access for end-users. Compared to state-of-art, \CRATE
support large number of users workiing on large shared
documents i.e. the number of users only impact communication
complexity. This main contributions of this paper are three-fold:
\begin{itemize}
\item we describe \CRATE's system and its core-components, namely \LSEQ and
  \SPRAY. \CRATE's Javascript implementation runs directly inside browsers
  without any plugins.
\item Compared to previous work~\cite{nedelec2013lseq}, we demonstrate the upper
  bound on space complexity of \LSEQ and we analyze the complete complexity of
  \CRATE compared to state of art.
\item We conducted experimental studies to validate the complexities of the
  \CRATE's implementation inside real web browsers. From $2$ to $600$
  participants editing a document reaching more than $1$ million characters, we
  observed a logarithmic growth of the traffic compared to the number of
  participants, and a polylogarithmic growth of the traffic compared to the size
  of the document.
\end{itemize}

The remainder of this paper is organized as
follows. Section~\ref{sec:relatedwork} reviews the related work with an emphasis
on the complexity tradeoff proposed by other
approaches. Section~\ref{sec:proposal} follows with a description of \CRATE and
its core components. Section~\ref{sec:experiments} highlights the scalability of
\CRATE while validating the space complexity analysis of \LSEQ and
\SPRAY. Finally, Section~\ref{sec:conclusion} concludes the paper and discusses
about perspectives.

%%% Local Variables: 
%%% mode: latex
%%% TeX-master: "../paper"
%%% End: 
