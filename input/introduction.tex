
\section{Introduction}

Distributed real-time editors such as Google Docs, or Etherpad allow
people to work distributed in space and across organizations. Millions of users
uses these editors. 

However, main stream real-time editors such Google Doc rely on centralized
servers. It implies issues in privacy, censorship and economic intelligence. It
also brings restrictions in terms of service. For instance, Google Docs
restricts the writing access of its documents to 50 users simultaneously.

On the other hand, decentralized real-time editors do not require any
collaboration providers. Many algorithms has been
proposed~\cite{sun1998operational, oster2006data, sun2009contextbased}, yet
finding a good compromise for a full working system remains challenging: the
holy Grail tradeoff between complexities in time, space and communications.
\TODO{Furthermore, competing with centralized tools means to be as accessible as
  them.}
%% Due to the
%% communication layer setup, It also difficult to achieve for a decentralized
%% editor a 'one-click' deployement as Google doc can do.

In this paper, we propose \CRATE, a real-time editor that runs on a network of
browsers. \CRATE relies on WebRTC and browser-to-browser communication to
provide an easy access for end-users. Compared to state-of-art, \CRATE proposes
a new complexity tradeoff between time, space, and communication that fits web
constraints in order to run within browsers.

This main contributions of this paper are three-fold:
\begin{itemize}
\item we describe \CRATE's system and its core-components, namely \LSEQ and
  \SPRAY. \CRATE's Javascript implementation runs directly inside browsers
  without any plugins.
\item Compared to previous work~\cite{nedelec2013lseq}, we demonstrate the upper
  bound on space complexity of \LSEQ and we analyze the complete complexity of
  \CRATE compared to state of art.
\item We conducted experimental studies to validate the complexities of the
  \CRATE's implementation inside real web browsers. From $2$ to $600$
  participants editing a document reaching more than $1$ million characters, we
  observed a logarithmic growth of the traffic compared to the number of
  participants, and a polylogarithmic growth of the traffic compared to the size
  of the document.
\end{itemize}

The remainder of this paper is organized as
follows. Section~\ref{sec:relatedwork} reviews the related work with an emphasis
on the complexity tradeoff proposed by other
approaches. Section~\ref{sec:proposal} follows with a description of \CRATE and
its core components. Section~\ref{sec:experiments} highlights the scalability of
\CRATE while validating the space complexity analysis of \LSEQ and
\SPRAY. Finally, Section~\ref{sec:conclusion} concludes the paper and discusses
about perspectives.

%%% Local Variables: 
%%% mode: latex
%%% TeX-master: "../paper"
%%% End: 
