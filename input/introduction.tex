
\section{Introduction}

Google Docs made real-time editing in browsers easy for millions of
users. However, Google mediates real-time editing sessions with central servers
raising issues on privacy, censorship, economic intelligence. It also raises
scalability issues in term of number of participants.  Despite that small groups
currently constitute the main range of users, events such as massive online
lectures, TV shows, conferences gather larger groups.  Google Docs supports
large groups but only the first fifty users can edit, next users have their
rights limited to document reading. \TODO{We think that real-time editors should
  allow editing at anytime and anywhere, whatever the number of participants.}
Real-time editing sessions can be highly dynamic; even if only few authors are
editing a document simultaneously, the editing session includes a much larger
group of potential writers. In this paper we focus on building a real-time
editor that supports privacy and that adapts \TODO{gently} from small groups to
large groups.

Decentralized real-time editors~\cite{oster2006data, sun1998operational,
  sun2009contextbased} do not require intermediate servers and by the same solve
privacy issues. However, scalability issues remain.  Addressing scalability
requires finding a good trade-off between communication, space and time
complexities. Achieving a sublinear communication complexity compared to the
editing session size is crucial for supporting large groups.  \TODO{In the best
  case, if we have $n$ participants, sending a typed character to others
  requires to contact $log(n)$ participants.} But consistency maintenance of
documents requires each message to piggyback additional information which
greatly impacts the communication complexity.

Decentralized algorithms of operational
transformation~\cite{sun2009contextbased} require piggybacking a state
vector in order to detect concurrent operations. Unfortunately, state
vectors grow linearly compared to the number of members that ever
participated in the authoring.

Conflict-Free Replicated Data Types~\cite{shapiro2011comprehensive}
(CRDTs) such as WOOT~\cite{oster2006data} piggyback a constant-size
identifier which is optimal. Nonetheless, their local space complexity
grows monotonically which eventually leads to the need of costly
garbage collecting. Other algorithms~\cite{preguica2009commutative,
  weiss2010logootundo} limit their local space
consumption. Nevertheless, they piggyback variable-size identifiers
that can grow linearly compared to the document size. In such case,
they eventually need to run a costly consensus algorithm to balance
documents~\cite{zawirski2011asynchronous}. Finally, Algorithm
\LSEQ~\cite{nedelec2013concurrency} aims to avoid such consensus by
sublinearly upper-bounding the space complexity of its identifiers. It
conjectured a polylogarithmic growth of the identifiers size
$\mathcal{O}((\log d)^2)$ where $d$ is the document size.

%% for positioning reason, table from related work is here
\begin{table*}[t]
  \centering
  

\begin{tabular}{@{}lccccccc@{}}
  \toprule
  & \multicolumn{4}{c}{\textsc{time}} & \multicolumn{2}{c}{\textsc{space}} & \textsc{communication} \\ \cmidrule{2-8}
  & \multicolumn{2}{c}{\textsc{local}} & \multicolumn{2}{c}{\textsc{remote}} & & & \\ \cmidrule{2-5}
  & \textsc{ins} & \textsc{del} & \textsc{ins} & \textsc{del} & \textsc{sequence} & \textsc{causality} & \textsc{message} \\ \midrule
  SOCT2/TTF & $O()$ & $O()$ & $O()$ & $O()$ & $O()$ & $O()$ & $O()$ \\ \midrule
  \TODO{COT-DO} & $O()$ & $O()$ & $O()$ & $O()$ & $O()$ & $O()$ & $O()$ \\ \midrule
  WOOT & $O()$ & $O()$ & $O()$ & $O()$ & $O()$ & $O()$ & $O()$ \\ \midrule
  WOOTO & $O()$ & $O()$ & $O()$ & $O()$ & $O()$ & $O()$ & $O()$ \\ \midrule
  WOOTH & $O()$ & $O()$ & $O()$ & $O()$ & $O()$ & $O()$ & $O()$ \\ \midrule
  RGA & $O()$ & $O()$ & $O()$ & $O()$ & $O()$ & $O()$ & $O()$ \\ \midrule
  \TODO{SW} & $O()$ & $O()$ & $O()$ & $O()$ & $O()$ & $O()$ & $O()$ \\ \midrule
  \TODO{PPS} & $O()$ & $O()$ & $O()$ & $O()$ & $O()$ & $O()$ & $O()$ \\ \midrule
  \TODO{Neil Conway} & $O()$ & $O()$ & $O()$ & $O()$ & $O()$ & $O()$ & $O()$ \\ \midrule
  Treedoc & $O()$ & $O()$ & $O()$ & $O()$ & $O()$ & $O()$ & $O()$ \\ \midrule
  Logoot & $O()$ & $O()$ & $O()$ & $O()$ & $O()$ & $O()$ & $O()$ \\ \midrule
  LogootSplit & $O()$ & $O()$ & $O()$ & $O()$ & $O()$ & $O()$ & $O()$ \\ \midrule
  \TODO{\LSEQ}  & $O()$ & $O()$ & $O()$ & $O()$ & $O()$ & $O()$ & $O()$ \\ \bottomrule
\end{tabular}


  \caption{\label{table:complexities}
    Communication and space complexities of decentralized approaches.
    $W$ is the number of writers, 
    $R$ is the number of replicas (readers and writers),
    $H$ is the number of operations in the historic (insertions and deletions),
    and $I$ is the number of insertions.
    Bottlenecks of each approach are highlighted.}
\end{table*}

In this paper, we propose \CRATE, a scalable real-time editor that runs on a
network of browsers. \CRATE relies on WebRTC and browser-to-browser
communication to provide an easy access for end-users. Compared to
state-of-the-art, it presents a new original trade-off which balances the load
between space, time, and communication complexities. The contributions of this
paper are threefold:
\begin{itemize}
\item Compared to previous work~\cite{nedelec2013lseq}, we demonstrate the upper
  bounds on space and time complexities of \LSEQ. It constitutes an original
  trade-off for building decentralized real-time editors.
\item \TODO{Compared to previous work~\cite{nedelec2015spray}}, we deploy the
  adaptive membership protocol \SPRAY in web browsers thanks to WebRTC. We
  observe that \SPRAY's adaptivity allows handling high dynamicity of real-time
  editing sessions.
\item We conducted experimental studies to validate the complexities of
  \CRATE. The experiments took place in the Grid'5000 testbed and involved
  uptill $600$ real web browsers opened to edit a shared document. At a rate of
  100 insertions per seconds, the document size reached above 1 million
  characters. As expected, we observed a logarithmic growth of the traffic
  compared to the number of participants, and a polylogarithmic growth of the
  traffic compared to the size of the document. This result confirms that \CRATE
  does not require any costly garbage collection mechanism.
\end{itemize}

The remainder of this paper is organized as follows:
Section~\ref{sec:relatedwork} reviews the related work with an emphasis on the
complexity trade-off proposed by other approaches. Section~\ref{sec:proposal}
follows with a description of \CRATE and its core
components. Section~\ref{sec:experiments} highlights the scalability of \CRATE
while validating the complexity analysis of \LSEQ and \SPRAY. Finally,
Section~\ref{sec:conclusion} concludes the paper and discusses about
perspectives.

%%% Local Variables: 
%%% mode: latex
%%% TeX-master: "../paper"
%%% End: 
