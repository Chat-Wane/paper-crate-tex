
\section{Introduction}

%Real-time collaborative editors are used by millions of users and allow them to
%distribute the work over locations and across organizations.


Google Doc or Etherpad allow millions of users to start a real-time
editing session by just clicking on a URL in their web
browsers. However, main real-time editors require collaboration
providers to setup servers to enable collaboration. This implies
issues in privacy, censorship, economic intelligence. It also implies
scalability issues in term of number of participants. If currently
real-time editors is mainly used by small groups, supporting large
groups occurs in contexts such as massive online courses,
conferences, TV-shows where people want to share their notes. Google
doc supports large groups but currently only fifty first users can
edit in real-time, next users can only view the shared document
without editing. We think that real-time editors should allow anybody
to edit anytime whatever the number of participants.

Decentralized real-time editors ~\cite{oster2006data,
  sun1998operational, sun2009contextbased} do not require intermediate
servers, solving some privacy issues. However, scalability issues remains.
%
% Deployement issues can be tackled with WebRTC. By allowing
% browser-to-browser data channels, WebRTC
% provides the opportunity to run decentralized editors in browsers with
% simple connection establishement. 

Addressing the scalabilty issue requires to find a good trade-off
between communication, space and time complexities. First, to be
scalable, communication complexity of decentralized editors has to be
sublinear to the number of participants. In the best case, if we have
$n$ participants, sending a typed character to others requires to
contact $log(n)$ participants. But consistency maintenance of
real-time editors requires to piggy back to each message extra
informations that impact greatly the whole communication complexity.

Decentralized algorithms of operational
transformation~\cite{sun2009contextbased} requires to piggy-back at
least a state vector in order to detect concurrent operations that is
linear to the number of participants.

Conflict-Free Replicated Data Type
(CRDT)\cite{shapiro2011comprehensive} such as
WOOT~\cite{oster2006data} piggy-back a fixed-size identifier which is
great but require unlimited space. Others algorithms can limit space
complexity, but piggy-back variable-size identifiers that can be linear to
the size of
document\cite{weiss2010logootundo,preguica2009commutative}. If the
worst case happens, the only solution is to run a costly consensus
algorithm to rebalance documents~\cite{zawirskiasynchronous}. Finally,
the LSEQ algorithm~\cite{nedelec2013concurrency} aims to avoid consensus
by bounding variable-size identifiers to a space complexity sublinear
to the size of shared document. It conjectured that identifiers can be
bounded to the $log(s)^2$ where $s$ is the size of shared document.

In this paper, we propose \CRATE, a scalable real-time editor that
runs on a network of browsers. \CRATE relies on WebRTC and
browser-to-browser communication to provide an easy access for
end-users. \CRATE belongs to Conflict-Free Replicated Datatypes (CRDT)
class of real-time editors. Compared to state of art, we reduced space
complexity of a shared document from linear to polylogarithmic
whithout changing time and space complexities of this class of editors.
 The contributions of this paper are three-fold:
\begin{itemize}
\item Compared to previous work~\cite{nedelec2013lseq}, we demonstrate
  the upper bound on space complexity of \LSEQ in case of monotonic
  editings and we analyze the complete complexity of \CRATE compared
  to state of art.
\item We describe \LSEQ for managing the shared document
  and \SPRAY a gossip algorithm based on WEBRTC. \CRATE's Javascript
  implementation runs directly inside browsers without any plugins.
\item We conducted experimental studies to validate the complexities of the
  \CRATE's implementation inside real web browsers. As expected, from $2$ to $600$
  participants editing a document reaching more than $1$ million characters, we
  observed a logarithmic growth of the traffic compared to the number of
  participants, and a polylogarithmic growth of the traffic compared to the size
  of the document.
\end{itemize}

The remainder of this paper is organized as
follows. Section~\ref{sec:relatedwork} reviews the related work with an emphasis
on the complexity tradeoff proposed by other
approaches. Section~\ref{sec:proposal} follows with a description of \CRATE and
its core components. Section~\ref{sec:experiments} highlights the scalability of
\CRATE while validating the space complexity analysis of \LSEQ and
\SPRAY. Finally, Section~\ref{sec:conclusion} concludes the paper and discusses
about perspectives.

%%% Local Variables: 
%%% mode: latex
%%% TeX-master: "../paper"
%%% End: 
