
\section{Introduction}

The recent WebRTC technology has opened a new playground for large-scale
distributed web applications. Among others, distributed collaborative editors
such as Google Docs, SubEthaEdit, or Etherpad, have proven their
usefulness. Such editors allow distributing the work across space, time and
organizations. However, they have several limitations due either to
centralisation (e.g. single-point-of-failure, privacy issues, economic
intelligence issues, limitations in terms of service) or to decentralisation
(e.g. number of users, concurrency and synchronisation, number of
operations). For instance Google~\cite{nichols1995high} allows only 50 users to
write a document together at a same time. These scalability issues preclude the
existence of a massive distributed collaborative editor.

Decentralised distributed collaborative editors use the optimistic replication
scheme \cite{johnson1975maintenance, saito2005optimistic} to provide high
availability and responsiveness. Locally, each user involved in the
collaboration owns a copy of the document (replica). Local operations are
executed and become effective locally before being broadcast. Remotely, each
user integrates the received operations to her replica. The convergence property
of optimistic replication states that replicas are allowed to temporarily
diverge. Yet, the resulting replicas become identical when all collaborators
have received all changes~\cite{demers1987epidemic}.  Thus, optimistic
replication requires a shared data type that ensures consistency and an
algorithm to disseminate changes.

Operational transformation~\cite{sun1998operational, sun2009contextbased} (OT)
approaches are the first techniques used to build distributed collaborative
editors. However, the time complexity of the integration of operations is
quadratically upper-bounded by the number of concurrent operations relatively to
the integrated one. Consequently, all collaborators suffer from few concurrent
operations making OT approaches non practical because of unpredictable latency.

On the other side, a family of distributed data structure called Conflict-free
Replicated Data Type~\cite{shapiro2011comprehensive, shapiro2011conflict} (CRDT)
has been introduced. A CRDT for sequences is a data structure well-suited for
building distributed collaborative editors that scale in presence of concurrent
operations. To ensure convergence, a unique and immutable identifier is
associated with each element (e.g. character, words, line, paragraph) in the
sequence (i.e. document). The set of identifiers is paired with a total order,
and this latter actually makes the sequence.  Nevertheless, these sequence data
structures have a space complexity issue. Indeed, they provide two update
operations (insert and delete) and use an identifier allocation function that
either depends on the type of the performed operation \cite{ahmed2011evaluating,
  conway2014language, grishchenko2010deep, oster2006data,
  preguica2009commutative, roh2011replicated, weiss2007wooki, wu2010partial,
  Yu2012stringwise}, or the arguments of the operation~\cite
{preguica2009commutative, andre2013supporting,weiss2009logoot}. In both cases,
the space overhead induced by the identifiers can be prohibitively high and
directly impacts performance. Unfortunately, this allocation is a crucial but a
non-trivial problem.

In this paper, we introduce \CRATE, a distributed and decentralized
collaborative editor running in web browsers. Its core components are
\LSEQ~\cite{nedelec2013concurrency, nedelec2013lseq}, an allocation strategy
which provides sub-linearly upper-bounded identifiers compared to the number of
insertions in the document, and \SPRAY (REF), an adaptive random peer sampling
protocol fitting the web application context. This paper comprises the following
contributions: 
% In this paper, we study sequence data structures \footnote{A short version of
%   this paper appeared in \cite{nedelec2013lseq,nedelec2013concurrency}.} at the
% core of which there is the identifier allocation function that provides the
% different collaborators with identifiers they assign to the elements of the
% associated sequence. This paper has four main contributions:
\begin{itemize}
\item An in-depth description of its core components.
\item A proof of the polylogarithmic space complexity of \LSEQ.
\item A experimental study that validates the space complexity of \LSEQ and the
  scalability of \CRATE. Parts of theses experiments ran on the Grid'5000
  testbed and involved up to $500$ emulated peers creating a document of one
  million characters at a global rate of one hundred operations per second.
\end{itemize}

The remainder of this paper is organised as
follows. Section~\ref{sec:preliminaries} provides the necessary background to
sequence data structures and the motivations. Section~\ref{sec:proposal}
follows with a description of \LSEQ and the proof of its space
complexity. Section \ref {sec:experiments} highlights the scalability of
\CRATE while validating the space complexity analysis of
\LSEQ. Section~\ref{sec:relatedwork} reviews related works. Finally,
Section~\ref{sec:conclusion} concludes the paper.

%%% Local Variables: 
%%% mode: latex
%%% TeX-master: "../paper"
%%% End: 
