
\parpic{\includegraphics[width=1in,clip,keepaspectratio]{img/brice.png}}
\noindent {\bf Brice N{\'e}delec}
received the MS degree in Software Architecture from the University of Nantes
(France) in 2012. He is currently a Ph.D. student at the University of Nantes as
a team member of GDD. His main research interests concern distributed
applications, collaborative editing, causality tracking,  eventual
consistency, and networks. \\ \\
\ \\ \ \\
\parpic{\includegraphics[width=1in,clip,keepaspectratio]{img/pascal.png}}
\noindent {\bf Pascal Molli}
graduated from Nancy University (France) and received his Ph.D. in Computer
Science from Nancy University in 1996. Since 1997, he is Associate Professor at
University of Nancy. His research topic is mainly Computer Supported Cooperative
Work, P2P and distributed systems and collaborative knowledge building. His
current research topics are: Algorithms for distributed collaborative systems,
privacy and security in distributed collaborative systems, and collaborative
distributed systems for the Semantic Web.
\\ 
\parpic{\includegraphics[width=1in,clip,keepaspectratio]{img/achour.png}}
\noindent {\bf Achour Mostefaoui} received his M.Sc.  in computer science in
1991, and a Ph.D. in computer science in 1994 both from the University of
Rennes. He has been Associate professor in the Computer Science of the
University of Rennes from 1996 to 2011 when he joined the University of Nantes
as full professor. He spent one semester in the Theory of Computing group of the
CSAIL Lab. in the Massachusetts Institute of Technology in 2007. Since September
2011, he is head of a Master's diploma in Computer Science (University of
Nantes) and is co-head of the GDD research team within the LINA Lab.  His
research interests are on distributed computing. A first direction concerns
agreement and calculability issues in asynchronous, fault-prone distributed
systems. More specifically, he is interested in discovering the boundary between
solvable and unsolvable tasks and, understanding further assumptions/relaxations
needed to solve otherwise impossible tasks. A second direction is to study
replicated date structure (queue, tuple space, transactionnal memory, wiki,
etc.). How to specify and implement distributed data structures and then
integrate them into programming languages in different distributed computing
models. For both directions, distributed algorithms design is the key point in
my work, either to study algorithmic mechanisms, to efficiently solve tasks, or
to show impossibility results through reductions.

