
\section{Decentralized real-time editor}
\label{sec:editor}

Distributed collaborative editors such as Google Docs, Etherpad, or SubEthaEdit,
allow collaborators to work distributed in time, space, and
organization~\cite{ellis1991groupware}. These tools improve the implication of
users as well as the quality of documents. However, trending editors such as
Google Docs rely on central servers which bring issues in privacy, scalability,
and robustness. Decentralized editors settle issues in privacy and robustness
but scalability issues remain.

Decentralized editors' architecture comprises 4 layers
(cf. Figure~\ref{fig:architecture}) and each of them can constitute a barrier to
scalability:
\begin{inparaenum}[(i)]
\item the communication layer includes the editing session membership mechanism
  and the information dissemination protocols;
\item the causality layer includes the causality tracking structure that
  guarantees a delivery order of operations reflecting a form of causality,
  e.g., it ensures that the removal of an element always follows its insertion;
\item the sequence structure layer includes a convergent replicated structure
  representing the document;
\item the graphical user interface includes the editor as a graphical entity
  that users can interact with. %% inside web browsers.
\end{inparaenum}

The left part of the figure depicts the common process chain: when the user
performs an operation on the document, the operation is applied to the
distributed sequence. Then it decorates the operation with causality tracking
metadata. Finally, the editor broadcasts it using the neighborhood provided by
the membership protocol.  Conversely, when the editor receives a broadcast
message, it checks if the operation is causally ready to be delivered. Once the
condition is verified, it applies the operation to the distributed sequence
which notifies the graphical user interface of the changes.  The right part of
the figure corresponds to the catch up strategy where a member may have missed
operations due to dropped messages, or simply because the user worked on offline
mode for a while. Therefore, it regularly asks to its neighborhood for the
missing operations.

\CRATE~\cite{nedelec2016crate} is a real-time decentralized editor running
directly within web browsers and available at
\url{https://github.com/Chat-Wane/CRATE}. None of its internal components
prevent its scalability. The rest of this section describes them. Among others,
we show that the communication complexity scales well in terms of number of
participants and document size. The rest of this section reviews each layer and
details the components inside them. Section~\ref{sec:experiments} validates its
scalability through experiments.

\begin{figure}
  \centering
  \begin{tikzpicture}[scale=0.95]

\newcommand\X{30pt}
\newcommand\Y{20pt}

\newcommand\LIGHTGRAY{gray!20}
\newcommand\MEDIUMGRAY{gray!40}

\small
%% communication
\draw[rounded corners=2mm, color=\MEDIUMGRAY, fill=white](0pt, 0pt)+(-4*\X,-\Y)rectangle+(4*\X,\Y);
\draw(4*\X, \Y)node[anchor=north east]{\textbf{communication}};

\draw[fill=white](-2*\X, -0.25*\Y)
node{broadcast}+(-0.75*\X,-0.5*\Y)rectangle+(0.75*\X,0.5*\Y);
\draw[fill=white, very thick]( 0*\X, 0.25*\Y)
node[align=center]{membership}+(-0.85*\X,-0.5*\Y)rectangle+(0.85*\X,0.5*\Y);
\draw[fill=white]( 2*\X, -0.25*\Y)
node{unicast}+(-0.75*\X,-0.5*\Y)rectangle+(0.75*\X,0.5*\Y);

\draw[<-](-0.85*\X, 0.25*\Y)--(-1.25*\X, -0.25*\Y);
\draw[<-](0.85*\X, 0.25*\Y)--(1.25*\X, -0.25*\Y);

%% causality
\draw[rounded corners=2mm, color=\MEDIUMGRAY, fill=\LIGHTGRAY](0pt, -2*\Y)+(-4*\X,-\Y)rectangle+(4*\X,\Y);
\draw(4*\X, -\Y)node[anchor=north east]{\textbf{causality}};

\draw[fill=\LIGHTGRAY](-2*\X, -2*\Y)
node[align=center]{\scalebox{0.77}{\shortstack{semantic\\causality\\tracker}}}
+(-1.0*\X,-0.6*\Y)rectangle+(1.0*\X,0.6*\Y);
\scriptsize
\draw[->, thick](-1.5*\X, -0.75*\Y) -- node[anchor=west]{receive}
(-1.5*\X, -1.4*\Y);
\draw[<-, thick](-2.5*\X, -0.75*\Y) -- node[anchor=east]{send}
(-2.5*\X, -1.4*\Y);
\small
\draw[<->]( 2*\X, -0.75*\Y)--( 1*\X, -2.5*\Y);

%% sequence structure
\draw[rounded corners=2mm, color=\MEDIUMGRAY, fill=white](0pt, -4*\Y)+(-4*\X,-\Y)rectangle+(4*\X,\Y);
\draw(4*\X, -3*\Y)node[anchor=north east, align=right]
{\textbf{sequence}\\\textbf{structure}};

\draw[fill=white, shading=axis,top color=\LIGHTGRAY, bottom color=white, shading angle=0](1*\X, -3*\Y)
node{anti-entropy}+(-0.95*\X,-0.5*\Y) rectangle +(0.95 *\X, 0.5*\Y);
\draw[fill=white, very thick](-2*\X, -4*\Y)
node{replica}+(-0.75*\X,-0.5*\Y) rectangle +(0.75 *\X, 0.5*\Y);

\draw[->] (0.05*\X, -2.75*\Y)--(-1*\X,-2*\Y);
\draw[->] (0.05*\X, -3.25*\Y)--(-1.25*\X,-4*\Y);
\scriptsize
\draw[<-, thick] (-1.5*\X, -3.5*\Y)--node[anchor=west]{deliver}(-1.5*\X, -2.6*\Y);
\draw[->, thick] (-2.5*\X, -3.5*\Y)--node[anchor=east]{decorate}(-2.5*\X, -2.6*\Y);
\small
%% gui
\draw[rounded corners=2mm, color=\MEDIUMGRAY, fill=\LIGHTGRAY](0pt, -6*\Y)+(-4*\X,-\Y)rectangle+(4*\X,\Y);
\draw(4*\X, -5*\Y)node[anchor=north east, align=right]
{\textbf{graphical}\\\textbf{user}\\\textbf{interface}};
\draw[fill=\LIGHTGRAY](0pt,-6*\Y)
node{web editor}+(-0.85*\X,-0.5*\Y) rectangle +(0.85 *\X, 0.5*\Y);

%%\draw[<->] (-2*\X, -4.5*\Y) -- (0*\X, -5.5*\Y);
\scriptsize
\draw[->, thick] (-1.80*\X, -4.5*\Y)--node[anchor=west]{notify}(-0.85*\X, -5.75*\Y);
\draw[<-, thick] (-2.20*\X, -4.5*\Y)--node[anchor=east]{update}(-0.85*\X, -6.25*\Y);
\small
\end{tikzpicture}
  \caption{\label{fig:architecture}The four layers of decentralized editors'
    architecture.}
\end{figure}

\subsection{Communication}
\label{subsec:communication}

Members of an editing session must be able to communicate their changes in a
scalable manner. Indeed, an editing session can gather from small to large
groups during its life time, e.g., massive open online courses (MOOC) may start
with a large number of students which can quickly decrease due to a lack of
interest. Also, editing sessions size depend of the targeted document, e.g., a
document describing a personal project and whose visibility is limited to
friends gathers significantly less people than a document about a large event,
such as a collaboratively written report about a conference.

The communication protocol must adapt to these change in membership
automatically. In that end, \CRATE uses \SPRAY which is a random peer sampling
protocol that incrementally builds the neighborhood of each editor.
Neighborhoods size scales logarithmically compared to the number of current
participants in the editing session. The broadcast protocol makes extensive use
of neighborhoods to propagate changes. Indeed, when a user performs a change,
the editor sends it to its neighbors. Each neighbor is in charge of integrating
and forwarding the change. Changes transitively reach all participants. The
communication complexity at each editor is $\mathcal{O}(m.\ln(R))$, where $m$ is
the message size, and $R$ is number of replicas connected during the
propagation.


% To collaboratively edit a document, users must establish a form of communication
% between them. It firstly requires accessing the editing session. It secondly
% requires building a network of communication channels. It thirdly requires the
% members to use it to spread the changes performed on shared documents.  

% \subsubsection{First access}

% The first access to an editing session transits a signaling server. A member
% wishing to share the document provides a \emph{uniform resource locator} (URL)
% that contains
% \begin{inparaenum}[(i)]
% \item an address to the \CRATE's files with the web application
%   code,
% \item and a reference to the editing session. Note that the latter must be
%   unique (for the signaling server and during the sharing time) and
%   immutable (to one document corresponds one editing session).
% \end{inparaenum}
% For instance, the URL \url{http://chat-wane.github.io/CRATE?snow-crab} targets
% the Github file server, and asks to the signaling server about the editing
% session called \emph{snow-crab}.  The signaling server chooses at random an
% available contact among sharers of this editing session. After a round-trip of
% messages, the first WebRTC connection is established, i.e., a connection from
% browser-to-browser that enables real-time communication. Since the joiner has
% access to the editing session through its contact, it does not need the
% signaling server any longer, hence, it disconnects from it. The member
% establishes other WebRTC connections using a membership protocol.

% \subsubsection{Membership}

% The membership protocol, called \SPRAY~\cite{nedelec2015spray}, is a random peer
% sampling protocol~\cite{jelasity2007gossip} the primary target of which is
% WebRTC.  As such, the range of users includes small devices (e.g. smartphones,
% tablets, etc.) and establishing a connection requires a three-way
% handshake. These constraints invite to maintain a small number of connections.

% Using \SPRAY, each member owns a set of neighbors which dynamically grows and
% shrinks to reflect the network size. Without any global knowledge,
% \begin{inparaenum}[(i)]
% \item it provides each member with a neighborhood of logarithmic size compared
%   to the global network size;
% \item it quickly converges to a topology exposing similarities with random
%   graphs~\cite{erdos1959random}. Among others,
%   \begin{inparaenum}[(a)]
%   \item it balances the load among members by repeatedly averaging over time the
%     size of neighborhoods pairwise;
%   \item it becomes robust to random crashes or unexpected departures of
%     members;
%   \item the shortest average distance to reach all peers stays small.
%   \end{inparaenum}
% \end{inparaenum}

% \SPRAY divides the life-cycle of a member into three phases: the joining, the
% exchanges, and the leaving. They respectively aim to increase, to retain, and
% to decrease the number of connections following a logarithmic progression.

% \subsubsection{Broadcast}

% The information dissemination protocol\cite{birman1999bimodal} aims to propagate
% the changes performed by users on their shared document. Any operation must
% reach all members (broadcast) to guarantee eventual consistency.  When a user
% performs an operation, \CRATE prepares a message including the result of the
% operation and sends it to the whole network using its neighborhood. Neighbors
% receiving such message forward it to their own neighbors. Hence, messages reach
% all participants transitively. To guarantee termination and to limit the
% flooding, each member forwards each message to their neigbhors only once by
% using a version vector with exceptions (cf. Section~\ref{subsec:causality}).

% Compared to state-of-the-art~\cite{ganesh2003peer, jelasity2007gossip,
%   voulgaris2005cyclon}, \SPRAY provides a neigbhorhood reflecting the network
% size instead of a constant size neighborhood set at start (commonly oversized to
% handle large networks). As such, information dissemination protocol on top of
% \SPRAY adapts the load to the network size.

% The information dissemination protocol impacts the communication complexity at
% each member:
% \begin{equation}
%   \mathcal{O}(m.\ln R)
% \end{equation}
% where $m$ is the message size determined by the layers below, and
% $R$ is the number of replicas in the network including both
% writers and readers of the shared sequence.


\subsection{Causality tracking}
\label{subsec:causality}

Causality tracking aims to track semantically related operations, e.g., the
removal of an element always follows its insertion. The more constraints on the
order of operation integration, the more costly it gets. Thus, accurately
tracking causality relations of one operation with all others requires at least
$\mathcal{O}(W)$ both locally and in communication overhead, where $W$ is the
number of participants that ever wrote in the document. Such communication
overhead confines its usage to small controlled editing session. \CRATE chooses
to track only the semantically related pairs of operations: the removal of an
elements with its insertion. If the operations arrives to an editor out of
order, the remove will wait for the corresponding insertion. On the opposite, it
immediately integrates received insertions. The structure is a version vector
with exceptions~\cite{malkhi2007concise, mukund2014optimized}. The vector
allocates for each member
\begin{inparaenum}[(i)]
\item an integer denoting the maximum counter of operations originated from
  this site and
\item a list of integers denoting the exceptions, i.e., the operations known
  as not received yet.
\end{inparaenum}
While the local overhead implied by such structure remains $\mathcal{O}(W)$, the
communication overhead is constant $\mathcal{O}(1)$.

The structure also serves as a tool to identify differences between replicas
when an editor needs to catch up with the current state of the document in the
live editing session. The anti-entropy protocol, whose \TODO{name literally
  means that its goal is to reduce the differences in information} between the
involved members, selects an editor from its neighborhood at random, and sends
its local version vector with exceptions. The chosen neighbor performs the
difference with its own vector, searches the missing operations, and sends them
back to the requesting editor. All editors periodically perform such
anti-entropy rounds to ensure that no operations went missing -- due to
unpredictable network behavior.

%% To guarantee the exactly once delivery of operations, and the causal delivery
%% of semantically related operations, \CRATE uses a version vector with
%% exceptions~\cite{malkhi2007concise, mukund2014optimized}.

% \subsubsection{Version vector with exceptions}

% A vector stores for each member
% \begin{inparaenum}[(i)]
% \item an integer denoting the maximum counter of operations originated from
%   this site and
% \item a list of integers denoting the exceptions, i.e., the operations known
%   as not received yet.
% \end{inparaenum}

% A unique member identifier along with a monotonically growing counter allows
% differentiating each operation. Thus, when a member performs a change to its
% shared document, it increments its local counter. Then it decorates the message
% with its counter and identifier. Upon reception, \CRATE checks in the version
% vector with exceptions if it already received the operation earlier. In this
% case, it simply discards the operation.  Otherwise, it checks if the operation
% depends on another one. In this case, \CRATE delivers the operation if this
% other operation is delivered. Otherwise, it puts the operation in a buffer
% awaiting for the dependency to arrive. Upon delivery, it integrates the
% operation identifier to the version vector with exceptions.

\begin{figure}
  \centering
  \begin{tikzpicture}[scale=0.965]
  
  \newcommand\X{ 20pt}
  \newcommand\Y{-30pt}

  \draw[->] (0pt,0pt) node[anchor=east]{$\pmb{m_1}$} -- (11*\X,0pt);
  \draw[->] (0pt,-30pt) node[anchor=east]{$\pmb{m_2}$} -- (11*\X,\Y);
  \draw[->] (0pt,-60pt) node[anchor=east]{$\pmb{m_3}$} -- (11*\X,2*\Y);

  \scriptsize
  \draw (0.5*\X,2+0*\Y) node[anchor=south]{[ ]} -- (0.5*\X,-2+0*\Y);
  \draw (0.5*\X,2+1*\Y) node[anchor=south]{[ ]} -- (0.5*\X,-2+1*\Y);
  \draw (0.5*\X,2+2*\Y) node[anchor=south]{[ ]} -- (0.5*\X,-2+2*\Y);

  \draw[->,densely dashdotted] (1.5*\X,0pt) -- (2.5*\X,2*\Y);
  \draw[->,densely dashdotted] (2.5*\X,0pt) -- (3.5*\X,1*\Y);
  \draw[->,densely dashdotted] (2.5*\X,0pt) -- (3.5*\X,2*\Y);


  \draw (5*\X,2+1*\Y) node[anchor=south]{[$\langle m_1,\,2,\,\pmb{\{1\}}\rangle$]}
  -- (5*\X,-2+1*\Y);
  \draw (5*\X,2+2*\Y) node[anchor=south]{[$\langle m_1,\,2,\,\varnothing\rangle$]}
  -- (5*\X,-2+2*\Y);

  \draw [->,densely dashdotted, very thick] (1.5*\X,0pt) to[out=35,in=135] (6.65*\X,0pt)
  to[out=-45,in=110] (7.5*\X,\Y);

  \draw[fill=white] (1.5*\X,0pt) circle (2pt);
  \draw[fill=white] (2.5*\X,0pt) circle (2pt);

  \draw[->,densely dashdotted] (6*\X,2*\Y)--(6.5*\X,1*\Y);
  \draw[->,densely dashdotted] (6*\X,2*\Y)--(6.5*\X,0pt);

  \draw[fill=black] (6*\X,2*\Y) circle (2pt);
  
  \draw[->,densely dashdotted, very thick]
  (6.5*\X,1*\Y)to[out=-35,in=-145]node[anchor=north]{\textbf{wait}}(8.5*\X,1*\Y);


  \draw (9.5*\X,2pt) node[anchor=south]
  {[$\langle m_1,\,2,\,\varnothing\rangle \langle m_3,\,1,\,\varnothing\rangle$]}
  -- (9.5*\X,-2pt);
  \draw (9.5*\X,2+1*\Y) node[anchor=south]
  {[$\langle m_1,\,2,\,\varnothing\rangle \langle m_3,\,1,\,\varnothing\rangle$]}
  -- (9.5*\X,-2+1*\Y);
  \draw (9.5*\X,2+2*\Y) node[anchor=south]
  {[$\langle m_1,\,2,\,\varnothing\rangle \langle m_3,\,1,\,\varnothing\rangle$]}
  -- (9.5*\X,-2+2*\Y);
  
  \begin{scope}[shift={(2*\X, 2.5*\Y)}]
  \draw[->,densely dashdotted] (0pt,-1pt) -- (10pt,-1pt)
  node[anchor=west]{messages};
  \draw[fill=white] (55pt,0pt)node[anchor=west]{$insert$}circle (2pt);
  \draw[fill=black] (92.5pt,0pt)
  node[anchor=west]{$delete(\langle m_1,\,1\rangle)$} circle (2pt);
  \end{scope}

\end{tikzpicture}

  \caption{\label{fig:timeline} Causality tracking example.}
\end{figure}

Figure~\ref{fig:timeline} depicts an editing session involving 3 users. The
version vector with exceptions starts empty. Member $m_1$ inserts two characters
in its document and broadcasts the corresponding messages. Member $m_3$ quickly
receives both operations. Since it did not receive these operations before, and
since they do not depend of any other operation, it integrates the operation
identifiers to its causality structure. It also delivers the operation to the
shared sequence structure. In the meantime, Member $m_2$ only receives the
second operation. Consequently, it marks the first operation of $m_1$ as
exception and still integrates the received operation. Then, $m_3$ removes the
first character inserted by $m_1$ and broadcasts it. While $m_1$ delivers the
removal immediately, $m_2$ waits since the targeted operation belongs to the
exceptions. Once it receives the missing first operation of $m_1$, the exception
disappears and the delete operation is performed.

% The local upper-bound on space complexity is:
% \begin{equation}
%   \mathcal{O}(W)
% \end{equation}
% where $W$ is the number of writers, i.e., users who modified the
% document at least once.  Such structure only requires to piggyback the
% identifiers of operation. Hence, the upper-bound on communication complexity
% is:
% \begin{equation}
%   \mathcal{O}(o.\ln R|)
% \end{equation}
% where $o$ is the operation size determined by the shared sequence structure
% (cf. Section~\ref{subsec:sequence}), and the rest is determined by the
% communication layer (cf. Section~\ref{subsec:communication}).

% \subsubsection{Anti-entropy} 

% The anti-entropy protocol periodically checks if the local replica diverges from
% another neighbor's one at random. It aims to retrieve missing operations that
% could have been lost during transmissions. For this purpose, a simple difference
% between vectors suffices.

% For instance, in the prior example depicted by Figure~\ref{fig:timeline}, Member
% $m_2$ receives the second operation of $m_1$ before its first. To catch up,
% $m_2$ could send its version vector with exceptions to one of its neighbors
% chosen at random. Assuming that it picks $m_3$, the latter detects that,
% compared to its own vector, the remote member missed the first operation of
% $m_1$. Then, it sends it back to $m_2$ along with its own vector. Finally, $m_2$
% follows the normal process for the received operations, and additionnaly merges
% its vector with the received one.

% Such protocol does not require any additional local metadata. However, the
% communication cost is prohibitively high which encourages to perform this
% reconciliation protocol with great care:
% \begin{equation}
%   \mathcal{O}(W+W+x.o)
% \end{equation}
% where the first $W|$ designates the vector contained in the initiating
% message, and $W+x.o$ the response to this message where $x.o$ are the missing
% operations.

\subsection{Shared sequence}

A shared sequence structure maintains consistent replicas of a document. Strong
eventual consistency~\cite{shapiro2011comprehensive} states that a system is
consistent iff replicas integrating an identical set of operations converge to
an equivalent state. Users visualize a same document.

Section~\ref{sec:proposal} describes the sequence structure representing
\CRATE's documents. Identifiers associated to each character allow retrieving
the sequence thanks to a total order. Thus, an insert operation creates a
unique, and immutable identifier. The editor sends the latter to all members of
the editing sessions. With the logarithmic factor of dissemination
(cf. Section~\ref{subsec:communication}), the resulting communication complexity
is expected to be inbetween $\mathcal{O}((\log I).(\ln R))$ and
$\mathcal{O}((\log I)^2.(\ln R))$ depending on the editing behavior, where $I$
is the number of insert operations performed on the sequence, and $R$ the number
of replicas in the editing session during the propagation.
    
\subsection{Graphical user interface}

To perform their changes on the document, users have a visual representation of
their respective local copy in their web browser.  Since CRDTs for sequences
provide functions relying on identifiers rather than indexes, \CRATE uses a
lookup function which gets the identifiers at a designated index, and converse.
Thus, when users perform their insertions, \CRATE gets the identifiers at the
targeted position and generates the new identifier. When it receives an
identifier, it firstly inserts it into the tree structure and secondly get its
index to notify the view. As such, users can edit in real-time, i.e. they do not
experience any freezes during their edition locally, remote insertions are
integrated as soon as possible.


% Each element in the tree keeps track of its number of children. Such information
% allows quick withdrawing of ranges of elements. Table~\ref{table:lseqlookup}
% shows the time complexity of the look-up operation. When the tree structure is
% balanced, only a small fraction of the tree need to be browsed to find the index
% of elements. On the other hand, the structure generated by monotonic editing
% leads to a linear look-up. Fortunately, the latter case does not happen at local
% insertions since \CRATE can lazily store the last generated identifier to create
% the new one. If a user starts to edit each time at different positions, the
% structure starts to be balanced, and the look-up becomes more efficient
% relatively to the document size. Upon reception, since users only have a partial
% view of the whole document, the look-up of the received operation may fall into
% a range of element not loaded by the view.


%%% Local Variables: 
%%% mode: latex
%%% TeX-master: "../paper"
%%% End: 
