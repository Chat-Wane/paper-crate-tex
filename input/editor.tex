
\section{Decentralized real-time editor}
\label{sec:editor}

Figure~\ref{fig:architecture} describes the decentralized editors' architecture
that comprises 4 layers where each can constitute an obstacle to scalability:
\begin{inparaenum}[(i)]
\item the communication layer includes the editing session membership mechanism
  and the information dissemination protocols;
\item the causality layer includes the causality tracking structure and the
  catch up mechanism; 
\item the sequence layer includes a convergent replicated structure representing
  the document;
\item the user interface layer includes the editor as a graphical entity that
  users can interact with.
\end{inparaenum}

\begin{figure}
  \centering
  \begin{tikzpicture}[scale=1.1]

\newcommand\X{25pt}
\newcommand\Y{20pt}

\newcommand\LIGHTGRAY{gray!20}

\small
%% communication
\draw[rounded corners=2mm, fill=white](0pt, 0pt)+(-4*\X,-\Y)rectangle+(4*\X,\Y);
\draw(4*\X, \Y)node[anchor=north east]{\textbf{communication}};

\draw[fill=white](-2*\X, -0.25*\Y)
node{broadcast}+(-0.75*\X,-0.5*\Y)rectangle+(0.75*\X,0.5*\Y);
\draw[fill=white, thick]( 0*\X, 0.25*\Y)
node{\SPRAY}+(-0.75*\X,-0.5*\Y)rectangle+(0.75*\X,0.5*\Y);
\draw[fill=white]( 2*\X, -0.25*\Y)
node{unicast}+(-0.75*\X,-0.5*\Y)rectangle+(0.75*\X,0.5*\Y);

\draw[<-](-0.75*\X, 0.25*\Y)--(-1.25*\X, -0.25*\Y);
\draw[<-](0.75*\X, 0.25*\Y)--(1.25*\X, -0.25*\Y);

%% causality
\draw[rounded corners=2mm, fill=\LIGHTGRAY](0pt, -2*\Y)+(-4*\X,-\Y)rectangle+(4*\X,\Y);
\draw(4*\X, -\Y)node[anchor=north east]{\textbf{causality}};

\draw[fill=\LIGHTGRAY](-2*\X, -2*\Y)
node[align=center]{version vector\\with\\exceptions}
+(-1.0*\X,-0.6*\Y)rectangle+(1.0*\X,0.6*\Y);

\draw[<->, thick](-2*\X, -0.75*\Y)--(-2*\X, -1.4*\Y);
\draw[<->]( 2*\X, -0.75*\Y)--( 1*\X, -2.5*\Y);

%% sequence structure
\draw[rounded corners=2mm, fill=white](0pt, -4*\Y)+(-4*\X,-\Y)rectangle+(4*\X,\Y);
\draw(4*\X, -3*\Y)node[anchor=north east, align=right]
{\textbf{sequence}\\\textbf{structure}};

\draw[fill=white, shading=axis,top color=\LIGHTGRAY, bottom color=white, shading angle=0](1*\X, -3*\Y)
node{anti-entropy}+(-0.95*\X,-0.5*\Y) rectangle +(0.95 *\X, 0.5*\Y);
\draw[fill=white, thick](-2*\X, -4*\Y)
node{\LSEQ}+(-0.75*\X,-0.5*\Y) rectangle +(0.75 *\X, 0.5*\Y);

\draw[->] (0.05*\X, -2.75*\Y)--(-1*\X,-2*\Y);
\draw[->] (0.05*\X, -3.25*\Y)--(-1.25*\X,-4*\Y);
\draw[<->, thick] (-2*\X, -3.5*\Y)--(-2*\X, -2.6*\Y);

%% gui
\draw[rounded corners=2mm, fill=\LIGHTGRAY](0pt, -6*\Y)+(-4*\X,-\Y)rectangle+(4*\X,\Y);
\draw(4*\X, -5*\Y)node[anchor=north east, align=right]
{\textbf{graphical}\\\textbf{user}\\\textbf{interface}};
\draw[fill=\LIGHTGRAY](0pt,-6*\Y)
node{web editor}+(-0.85*\X,-0.5*\Y) rectangle +(0.85 *\X, 0.5*\Y);

%%\draw[<->] (-2*\X, -4.5*\Y) -- (0*\X, -5.5*\Y);
\draw[<->, thick] (-2*\X, -4.5*\Y) -- (-0.85*\X, -6*\Y);
\end{tikzpicture}
  \caption{\label{fig:architecture}The four layers of decentralized editors'
    architecture.}
\end{figure}


The left part of the figure depicts the common process chain: when a user
performs an operation on the document, the operation is applied to the
distributed sequence. Then it decorates the operation with causality tracking
metadata. Finally, the editor broadcasts it using the neighborhood provided by
the membership protocol.  Conversely, when the editor receives a broadcast
message, it checks if the operation is causally ready to be delivered. Once the
condition is verified, it applies the operation to the distributed sequence
which notifies the graphical user interface of the changes.  The right part of
the figure corresponds to the catch up strategy where a member may have missed
operations due to dropped messages, or simply because the user worked offline
for a while. Therefore, when the editor is online, it regularly asks to its
neighborhood for the missing operations.

\CRATE~\cite{nedelec2016crate} is a real-time decentralized editor running
directly within web browsers and available at
\url{https://github.com/Chat-Wane/CRATE}. 
This section describes the model of \CRATE: communication, causality, and
sequence.  It reviews each component and its contribution to the system. This
composition provides a tradeoff favoring communication complexity. The system
scales well in terms of number of participants and document
size. Section~\ref{sec:experiments} validates this claim through experiments.



\paragraph{Communication.}
Editing sessions can gather from small to large groups during their life time,
e.g., massive open online courses (MOOC) may start with a large number of
students which can quickly decrease due to a lack of
interest~\cite{breslow2013studying}. Also, editing sessions vary from their size
according to the scope of documents. For instance, a document describing a
personal project and the visibility of which is limited to friends gathers
significantly less people than a document about a large event, such as a
collaboratively written report about a conference. The communication layer
should transparently handle any editing session size in a scalable manner.


\CRATE uses \SPRAY~\cite{nedelec2015spray} to automatically adapt its
functioning to the editing session size. \SPRAY is a random peer sampling
protocol~\cite{jelasity2007gossip} that incrementally builds and maintains the
neighborhood of editors. Thus, each editor has a small set of editors to
communicate with. The size of sets scales logarithmically compared to the number
of current participants in the editing session. If the editing session starts
with 10 participants, they have 2.3 neighbors in average. If the editing session
grows to 1000 connected participants, they have 6.9 neighbors in average. If the
editing session shrinks to 10 participants again, they have 2.3 neighbors in
average again.

To broadcast operations, the editor makes extensive use of neighborhoods to
propagate changes. Indeed, when a user performs a change, the editor sends it to
the editors in its neighborhood. Each neighbor is in charge of integrating and
forwarding the change to its own neighbors. Changes transitively reach all
editors very quickly, for the average shortest path from an editor to others
remains very small. The communication complexity at each editor is upper-bounded
by $\mathcal{O}(M \ln(R))$, where $M$ is the message size, and $R$ is number of
replicas connected during the propagation.

\paragraph{Causality tracking.}

To preserve consistent replicas, the same outcome must result from the
generation of the operation and its integration~\cite{sun1998achieving}. Often,
the behavior of an operation depends of others previously integrated. For
instance, the generation of a removal operation requires the targeted element to
exist, hence, its insert operation to be integrated. These \emph{happens before}
relationships~\cite{lamport1978time} constrain the integration order.
Unfortunately, the more the order is constrained, the costlier it
becomes. Accurately tracking causal relations of one operation with all others
requires at least $\mathcal{O}(W)$ both locally and in communication overhead,
where $W$ is the number of participants that ever wrote in the
document~\cite{charronbost1991concerning}. Such communication overhead confines
its usage to small editing sessions.

\CRATE uses a version vector with exceptions~\cite{malkhi2007concise,
  mukund2014optimized}. Each operation is uniquely identified by a unique site
identifier along with a counter. Each broadcast message includes such
identifier. For each editor,
\begin{inparaenum}[(i)]
\item an integer denotes the maximal counter of received operations that
  originated from this editor and
\item a set of integers denotes the exceptions, i.e., the operations known as
  not yet received from this editor.
\end{inparaenum}

This causality tracking structure tracks only the semantically related pairs of
operations (e.g. the removal of an element with its insertion). If the
operations arrives to an editor out of order, the removal waits for the
corresponding insertion. On the opposite, it immediately integrates received
insertions.
The structure also serves as a tool to identify differences between replicas
when an editor needs to catch up with the current state of the document in the
live editing session.  Each editor periodically performs an
anti-entropy~\cite{demers1987epidemic} round ensuring that no operations went
missing due to an unreliable network or offline writing.

While the local overhead implied by such structure is upper-bounded by
$\mathcal{O}(W)$, the communication overhead is constant $\mathcal{O}(1)$.

\paragraph{Shared sequence.}

Sun et al. state that collaborative editing requires convergence, causality, and
intention preservation~\cite{sun1998achieving}.
A replicated structure for sequences aims to provide eventually
convergent~\cite{shapiro2011conflict} copies of the document while providing
meaningful operations preserving intention. Thus, users visualize the same
document and can modify it consistently.

\CRATE uses a conflict-free replicated data type for
sequences~\cite{shapiro2011conflict} to represent its documents. Such sequence
types rely on unique and immutable identifiers. \CRATE uses \LSEQ to provide the
identifiers. Delete operations truly remove the targeted elements from the
underlying structure. Furthermore, the size of these identifiers are expected to
be bounded between a logarithmic bound and a polylogarithmic bound compared to
the number of insertions in the sequence.
Being sublinear, the structure does not require balancing which needs consensus
algorithms that do not scale~\cite{mostefaoui2015signature}.

\paragraph{Communication complexity.}

The logarithmic multiplicative factor of broadcast, the constant cost of
causality tracking, and the cost of identifiers, lead to a communication
complexity bounded between $\mathcal{O}((\log I).(\ln R))$ and
$\mathcal{O}((\log I)^2.(\ln R))$ depending on the editing behavior, where $I$
is the number of insert operations performed on the sequence, and $R$ the number
of replicas in the editing session during the message propagation.

%%% Local Variables: 
%%% mode: latex
%%% TeX-master: "../paper"
%%% End: 
